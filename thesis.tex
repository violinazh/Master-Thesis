%% LaTeX2e class for student theses
%% thesis.tex
%% 
%% Karlsruhe Institute of Technology
%% Institute for Program Structures and Data Organization
%% Chair for Software Design and Quality (SDQ)
%%
%% Dr.-Ing. Erik Burger
%% burger@kit.edu
%%
%% See https://sdq.kastel.kit.edu/wiki/Dokumentvorlagen
%%
%% Version 1.3.6, 2022-09-28

%% Available page modes: oneside, twoside
%% Available languages: english, ngerman
%% Available modes: draft, final (see README)
\documentclass[oneside, english]{sdqthesis}

%% ---------------------------------
%% | Information about the thesis  |
%% ---------------------------------

%% Name of the author
\author{Violina Zhekova}

%% Title (and possibly subtitle) of the thesis
\title{Detecting Ambiguity in Neural Machine Translation Models by Inspecting Diversity in Translation}

%% Type of the thesis 
\thesistype{Master's Thesis}

%% Change the institute here, ``KASTEL'' is default
\myinstitute{Artificial Intelligence for Language Technologies (AI4LT)}

%% You can put a logo in the ``logos'' directory and include it here
%% instead of the SDQ logo
% \grouplogo{myfile}
%% Alternatively, you can disable the group logo
\nogrouplogo

%% The reviewers are the professors that grade your thesis
\reviewerone{Prof. Dr. Jan Niehues}
\reviewertwo{Prof. Dr. Alexander Waibel}

%% The advisors are PhDs or Postdocs
\advisorone{M. Sc. Tu Anh Dinh}
%\advisortwo{}

%% Please enter the start end end time of your thesis
\editingtime{01. May 2023}{02. November 2023}

\settitle


%% --------------------------------
%% Additional packages                 |
%% --------------------------------

% Subfigures
%\usepackage{subfloat}
%\usepackage{subfigure}
%\usepackage{subfig}
\usepackage{subcaption}

% Table with automatic equal width columns spanning the page width
\usepackage{tabularx}

% Vertical table
\usepackage{pdflscape}

% Table spanning pages
%\usepackage{ltablex}
\usepackage{longtable}
% Brings the functionalities of longtable to tabularx
\usepackage{xltabular}

\usepackage{multirow}
\usepackage{makecell}


% Package for hypothesis
\usepackage[hyperref]{ntheorem}
\usepackage{thmtools}
\newtheorem*{hyp}{Hypothesis} 
\newtheorem*{subhyp}{Subhypothesis}
   \renewcommand\thesubhyp{\thehyp\alph{subhyp}}


% % *************** Hypothesis Labeling ***************
% \newtheorem{hypoin}{Hypothesis}[chapter]
% \newtheorem{hypoaux}{Hypothesis}[chapter]
% \usepackage{environ,etoolbox}
% \makeatletter
% \def\addtohypolist{\addtheoremline{hypoaux}}
% \NewEnviron{hypo}{%
%   \refstepcounter{hypoaux}%
%   \expandafter\addtohypolist\expandafter{\expandafter\ignorespaces\BODY}
%   \begin{hypoin}\BODY\end{hypoin}
% }
% \patchcmd\listtheorems
%   {\begingroup}
%   {\begingroup\let\label\@gobble}
%   {}{}
% \makeatother

% \usepackage{cleveref}
% \crefname{hyp}{Hyp}{Hyp} % always uppercase
% \crefname{subhyp}{Subhyp}{Subhyp} % always uppercase


\usepackage{pdfpages}

%% --------------------------------
%% | Bibliography                 |
%% --------------------------------

%% Use biber instead of BibTeX, see README
% style=authoryear; without numbers in Bibliography
\usepackage[style=numeric, citestyle=authoryear, maxbibnames=15, maxcitenames=2, abbreviate=false, backend=biber, natbib=true]{biblatex}
\addbibresource{thesis.bib}

%% ====================================
%% ====================================
%% ||                                ||
%% || Beginning of the main document ||
%% ||                                ||
%% ====================================
%% ====================================
\begin{document}

%% Set PDF metadata
\setpdf

%% Set the title
\maketitle

%% The Preamble begins here
\frontmatter

%\thispagestyle{empty}
\null\vfill
\noindent\hbox to \textwidth{\hrulefill} 
\iflanguage{english}{I declare that I have developed and written the enclosed
thesis completely by myself. I have submitted neither parts of nor the complete 
thesis as an examination elsewhere. I have not used any other than the aids that
I have mentioned. I have marked all parts of the thesis that I have included from 
referenced literature, either in their original wording or paraphrasing their
contents. This also applies to figures, sketches, images and similar depictions,
as well as sources from the internet.}%
{Ich versichere hiermit, dass ich die vorliegende Arbeit selbstständig verfasst,
und weder ganz oder in Teilen als Prüfungsleistung vorgelegt und keine anderen
als die angegebenen Hilfsmittel benutzt habe. Sämtliche Stellen der Arbeit, die
benutzten Werken im Wortlaut oder dem Sinn nach entnommen sind, habe ich durch
Quellenangaben kenntlich gemacht. Dies gilt auch für Zeichnungen, Skizzen,
bildliche Darstellungen und dergleichen sowie für Quellen aus dem Internet. }
 
 
%% ---------------------------------------------
%% | Replace PLACE and DATE with actual values |
%% ---------------------------------------------
\textbf{PLACE, DATE}
\vspace{1.5cm}
 
\dotfill\hspace*{8.0cm}\\
\hspace*{2cm}(\theauthor) 
\cleardoublepage
\includepdf{Declaration.pdf}

\setcounter{page}{1}
\pagenumbering{roman}

%% ----------------
%% |   Abstract   |
%% ----------------
 
%% For theses written in English, an abstract both in English
%% and German is mandatory.
%%
%% For theses written in German, a German abstract is sufficient.
%%
%% The text is included from the following files:
%% - sections/abstract

\includeabstract

%% ------------------------
%% |   Table of Contents  |
%% ------------------------
\tableofcontents

\listoffigures
\listoftables

%% -----------------
%% |   Main part   |
%% -----------------

\mainmatter

\chapter{Introduction}
\label{ch:Introduction}

% Title: Detecting Ambiguity in Neural Machine Translation Models by Inspecting Diversity in Translation

% PRICoBE:
% Problem: What is the problem you are trying to address?
% Research questions: What are the research questions you are trying to answer? (to make novelty/originality clear)
% Idea: What is your idea of how to address the problem? (is often broader than the contribution)
% Contributions: Which contributions to the research area are you making with that idea?
% Benefits: What are the benefits of your contributions?
% Evaluation: How do you plan to evaluate your envisioned benefits? (Validation)

% Define clear research objectives: Clearly define the goals and objectives of your research. These objectives should be specific, measurable, achievable, relevant, and time-bound (SMART).


% Introduce ML, MT, NMT
In the world more than 7000 natural languages are spoken nowadays. It is humanly impossible to learn every language, which outlines the need for translation between different languages for the purpose of communication. This is a task for Machine Translation (MT), a subfield of Machine Learning (ML) that focuses on automatic translation from one language to another using computer technology. Machine Translation (MT) technology has assisted humans in gathering, processing, and communicating information.  

%%%%%%%%%%%%%%%%%%%%%%%%%%%%%%%%%%%%%%%%%%%%%%%%%%%%%%%%%%%%%%%%%%%%%%%%%%%%%%%%%%%%%%%%%%%%
\section{Motivation}
\label{sec:Introduction:Motivation}

% Elaborate on the problem of bias and ambiguity (interplay of bias and ambiguity) and how it affects different individuals
The development of Artificial Intelligence (AI) in recent years has expanded the field of MT and made it possible for people from all over the world to connect, learn and work in a foreign language. One application of AI is Neural Machine Translation (NMT), which uses deep learning to learn a statistical neural model for machine translation in an end-to-end fashion, translating directly from an input source language to an output target language. Neural networks (NNs) are typically trained on large corpora of natural occurring data extracted from the internet \parencite{NMT}. 
One problem with this data is it often contains social constructs and stereotypes. As a consequence, NMT models learn the biases from the data and perpetuate them, affecting downstream applications like coreference resolution \parencite{Zhao_2018_coreference} and contributing further to discrimination based on gender, race, age and religious beliefs \parencite{Rudinger_2017}. Some examples of this phenomenon include under-representation of women, stereotyping professions, e.g., associating doctors with men and nurses with women \parencite{Escud_Font_2019}, and stereotyping behaviors, e.g., associating women with gossiping and men with guitars \parencite{Rudinger_2017}. Stereotypical assumptions in turn tend to impact individuals' perceptions of reality and influence their behavior in accordance with stereotypical expectations.

%%%%%%%%%%%%%%%%%%%%%%%%%%%%%%%%%%%%%%%%%%%%%%%%%%%%%%%%%%%%%%%%%%%%%%%%%%%%%%%%%%%%%%%%%%%%
\section{Research Questions}
\label{sec:Introduction:Questions}

% State objective
The objective of this work is to develop a method to detect ambiguous words in written text by inspecting the diversity of translation. In order to achieve this, I attempt to answer the following questions systematically.

\paragraph{Main Research Question: } How can we detect ambiguous words in translated written text?
\begin{itemize}
    \item How diverse are translations? % inspecting diversity patterns in translations which could point to ambiguity
    \item How does language influence the discovery of ambiguity? % different language families and alphabets may have different effect on translation ambiguous words
    \item How does context influence the discovery of ambiguity? 
\end{itemize}

% TODO: mention the type of ambiguity I am detecting (unresolved ambiguity, gender bias)

%%%%%%%%%%%%%%%%%%%%%%%%%%%%%%%%%%%%%%%%%%%%%%%%%%%%%%%%%%%%%%%%%%%%%%%%%%%%%%%%%%%%%%%%%%%%
\section{Contribution}
\label{sec:Introduction:Contribution}

As a part of the thesis, I want to contribute to solving the problem of bias in Machine Translation by developing an approach for detecting ambiguous words that could lead to bias. 
% TODO: Summarize approach, evaluation method and results 

In recent years, light has been shed on the different types of biases present in Neural Machine Translation (NMT) systems, the most researched type being gender bias \parencite{Savoldi_2021}. Some previous works have attempted to uncover gender bias in existing systems \parencite{Prates_2019}, while others have tried mitigating gender bias by either modifying the data (e.g, \citet{Escud_Font_2019}, \citet{Stanovsky_2019}) or changing the architecture of the system \parencite{Vanmassenhove_2018}. While there have been some studies on finding biases in MT, this is the first study aiming to create a framework for detecting ambiguity in a text, which contains no contextual information relating to the ambiguity, therefore making several translations possible. The ability to uncover ambiguity could in turn help to alleviate the problem of MT systems making an unjustified assumption, leading to bias.


%%%%%%%%%%%%%%%%%%%%%%%%%%%%%%%%%%%%%%%%%%%%%%%%%%%%%%%%%%%%%%%%%%%%%%%%%%%%%%%%%%%%%%%%%%%%
\section{Thesis Outline}
\label{sec:Introduction:Outline}
The rest of the thesis is structured as follows. Chapter \ref{ch:Background} describes the background of NMT systems and introduces the problems stemming from language ambiguity and bias in the data. Next, Chapter \ref{ch:Related_work} introduces some existing research on the topic, such as techniques to detect, assess, and mitigate bias in MT. Chapter \ref{ch:Methodology} states the research problem and describes the approach used to answer the research questions. Furthermore, Chapter \ref{ch:Setup} describes the design of the experiments performed, which includes the corpora, models and evaluation methods used to conduct the experiments as well as technical details necessary for reproducibility. Chapter \ref{ch:Experiments} describes the experiments, which were performed. In Chapter \ref{ch:Results} the results of the experiments are presented and discussed. Chapter \ref{ch:Discussion} outlines the answers to the research questions from conducting the experiments and discusses challenges and limitations. Finally, Chapter \ref{ch:Conclusion} summarizes the key findings and proposes possible directions for future work.
\chapter{Background}
\label{ch:Background}

In this chapter, I present the concepts relevant to the subject of the thesis. First, I introduce the topic of MT and the type of architecture I focus on later in the work. Next, I outline the problem of ambiguity and bias in MT models.

% WSD (if I end up using it)
% QE (if I end up using it)

%%%%%%%%%%%%%%%%%%%%%%%%%%%%%%%%%%%%%%%%%%%%%%%%%%%%%%%%%%%%%%%%%%%%%%%%%%%%%%%%%%%%%%%%%%%%
\section{Neural Machine Translation}
\label{sec:Background:NMT}

Machine Translation (MT) is the process of using computer technology to translate text from one natural language to another. This can be achieved using different paradigms. There are three main types of machine translation systems: Rule-based Machine Translation (RBMT), Statistical Machine Translation (SMT) and Neural Machine Translation (NMT). 

Conventional RBMT systems use pre-defined rules based on syntax, morphology and semantics, created by professional linguists. Since language is dynamic and evolves over time, these rules need frequent adaptation, which is costly. However, the key weakness of rule-based translation systems is that they require extensive lexicons and a large set of rules \parencite{SMT_book}. 

SMT systems, on the other hand, use a data-driven approach that utilizes statistical models derived from the analysis of bilingual and monolingual corpora. The quality of SMT output depends heavily on the size and quality of the corpora used to train the models. SMT’s general weakness is that it can only translate a phrase if it exists in the training dataset \parencite{SMT_book}.

Neural Machine Translation (NMT) is a subfield of SMT, which uses an artificial neural network to learn a statistical model for machine translation. Unlike traditional SMT systems, which require a pipeline of specialized components such as language model and translation model, NMT trains its statistical model end-to-end, mapping directly from an input source language to an output target language. NMT can recognize patterns in the training data to determine a context-based interpretation that can predict the likelihood of a sequence of words. Unlike SMT, NMT models are able to learn from each translation task and improve upon each subsequent translation. NMT models are more memory-efficient and also have a higher accuracy than SMT models, which makes them the appropriate choice for creating high-quality MT systems \parencite{NMT_book}.

% TODO: How in deep should I explain the Transformer and NMT?

%%%%%%%%%%%%%%%%%%%%%%%%%%%%%%%%%%%%%%%%%%%%%%%%%
\subsection{Sequence-to-Sequence Modeling}
\label{sec:Background:Seq2Seq}
The task of NMT is typically solved using Sequence-to-Sequence (Seq2Seq) modeling \parencite{seq2seq}. A Seq2Seq model has two parts: an encoder and a decoder. Both work separately and come together to form a large neural network model. This architecture has the ability to handle input and output sequences of variable length. A simplification of the architecture of NMT models can be seen in Fig. \ref{fig:seq2seq}. Firstly, each word in the input sentence is fed separately into the encoder to encode the source sentence into an internal fixed-length representation called the context vector. This context vector contains the meaning of the sentence. Secondly, the decoder decodes the fixed-length context vector and then predicts the output sequence.

\begin{figure}
  \centering
  \includegraphics[scale=0.57]{figures/seq2seq.png}
  \caption{Sequence-to-Sequence Modeling}
  \label{fig:seq2seq}
\end{figure}

The original architecture consists of a pair of Recurrent Neural Networks (RNNs) in the roles of encoder and decoder. RNNs process the input sequence token by token, which prohibits parallelization and makes the training and inference slow, especially when processing longer sequences. Also, they suffer from vanishing or exploding gradients, which is inconvenient for effective training. One solution for these problems served Long Short-Term Memory (LSTM) networks, a type of RNN that has additional memory gates to regulate the flow of information through the network better \parencite{lstm}. Despite this, using a fixed-length context vector still incurs a bottleneck in the model. To alleviate this problem, the use of attention-based architectures for neural machine translation was explored \parencite{attention}. 

The attention mechanism allows the decoder to look at the source tokens that are relevant while generating the next token. Despite all these efforts, using RNN-based encoder and decoder still forces the network to handle input sequentially, which makes it difficult to handle long-range dependencies within the input and output sequences from memory. Hence, \cite{transformer} proposed the Transformer architecture, which replaces RNNs with self-attention layers in the Encoder-Decoder network. Since in this work I make use of models based on the Transformer, next I will introduce its basic principle and components.

%%%%%%%%%%%%%%%%%%%%%%%%%%%%%%%%%%%%%%%%%%%%%%%%%
\subsection{Transformer Architecture}
\label{sec:Background:Transformer}
A Transformer is a Seq2Seq model, introduced by \cite{transformer}. An important feature of the Transformer architecture is its attention mechanism. The attention module looks at an input sequence and decides at each step which other parts of the sequence are important, differentially weighting the significance of each part of the input data. Like RNNs, Transformers are designed to handle sequential input data, such as natural language. However, unlike RNNs, Transformers can process the whole input sequence in parallel. The attention mechanism provides context for any position in the input sequence. This feature allows for more parallelization than RNNs and therefore reduces training times significantly \parencite{transformer}.

% Transformer Components
The Transformer architecture as presented in the original paper by \cite{transformer} is depicted in Fig. \ref{fig:transformer}.
The input embedding layer converts the high-dimensional input sequence into a low-dimensional sequence of vectors to capture the meaning and context.
The positional encoding preserves the sequential order of words in the input sentence and can be thought of as the distance of one word to another word in a sequence. This relative position of the words in the sequence is needed since the words are passed in parallel, as opposed to RNNs, which process them in order.
Self-attention is the weighted sum of all other words in the input sequence for each word using similarity (dot product) and SoftMax to focus on the most relevant parts of the input for each element. The multi-head attention repeats self-attention multiple times based on how many encoder/decoder layers there are.

There are multiple encoder and decoder layers. 
Each encoder has one multi-head self-attention, which encodes the weight of the input words to each other. 
Each decoder has one masked multi-head self-attention and one multi-head attention. The masked multi-head self-attention ensures that only words coming before a word are compared to that word, which means it only attends to preceding words in the input sequence during the decoding process. Applying a mask forces the model to ignore future words and focus only on the preceding words during the attention computation. % The mask specifically sets the attention weights to a very large negative value for the positions that correspond to future words, effectively making those weights close to zero after applying the softmax function. 
The multi-head cross-attention module in the decoder compares output tokens to input tokens.
Both the encoder and the decoder have one feed-forward layer, as well as adding and normalizing of residuals at each stage after the attention layers.

The output from the decoder is a vector of length of the input tokens. This output is fed into a fully connected (linear) layer to map it to a set of output prediction and then converted into probability over possible words like multi-class classification using a SoftMax layer.


The Transformer revolutionized NMT by replacing recurrence with attention, which allowed for simultaneous computations and more effective handling of long-range dependencies. This makes them efficient on hardware like GPUs and TPUs and pushes them to be the rational choice for architecture in the realm of MT.

\begin{figure}
  \centering
  \includegraphics[scale=0.5]{figures/transformer.png}
  \caption{The Transformer Architecture}
  \label{fig:transformer}
\end{figure}

% Explain autoregressive sampling?

%%%%%%%%%%%%%%%%%%%%%%%%%%%%%%%%%%%%%%%%%%%%%%%%%%%%%%%%%%%%%%%%%%%%%%%%%%%%%%%%%%%%%%%%%%%%
\section{Ambiguity and Bias in Machine Translation}
\label{sec:Background:Ambiguity_Bias}
Biases present in AI systems are an important problem stemming from cultural and historical issues present in the data from which models are learning. The developed systems in turn reinforce the present societal prejudices and old social norms, instead of mitigating them. 
It is important to understand how these biases occur in translation and to differentiate the different types of bias one may face.

Next, I will define the concepts of ambiguity and bias.

\paragraph{Ambiguity}
Ambiguity refers to the quality of being open to more than one interpretation, as in not having one obvious meaning. It is the type of meaning in which a phrase, statement, or resolution is not explicitly defined, making several interpretations plausible. A common aspect of ambiguity is uncertainty. In MT, ambiguity occurs when the source text leaves some essential properties unspecified, but the target language requires the property to be specified for correct translation. 

The ambiguity can be \textbf{resolvable} or \textbf{unresolvable}. It is resolvable when some semantic property required for the subject to be disambiguated can be found in the context, which defines the rest of the text available to the translation system. On the other hand, it is unresolvable, when no property necessary for disambiguation can be inferred from the context. To illustrate these two cases, we will look at two examples. When translating the sentence "She is a doctor." from English to German, which has no gender-neutral word for "doctor", the translation system has to choose the male ("Arzt") or female ("Ärztin") gender word for "doctor". In this case, the word "doctor" is ambiguous. However, the gender is resolvable from context due to the presence of the female pronoun "she". In contrast, the example sentence "I am a doctor." also contains the ambiguous word "doctor", but it is not indicated in the rest of the text whether the intended referent of "I" and "doctor" is a man or a woman. This makes the ambiguity in this case unresolvable \parencite{bias_taxonomy}.

When the ambiguity is unresolvable, the translation system cannot make an informed decision and instead applies randomness or previously acquired knowledge in choosing the translation, making an \textbf{unjustified assumption}. The assumption is unjustified because nothing actually present in the source text justifies it. In the example of "I am a doctor." the translator typically decides for the male translation of the word "doctor", because this case appears most often in similar contexts in its training data. Although, context allows for two possible translations in German for the ambiguous word "doctor", the system will consistently prefer the male translation, which leads to a \textbf{bias}. 

\paragraph{Bias}
Bias refers to a disproportionate weight in favor of or against an idea or thing, usually in a way that is closed-minded, prejudicial, or unfair. In machine translation, it is the tendency to make certain unjustified assumptions more often than others. Biases stem from unresolved ambiguities leading to unjustified assumptions and can be categorized as follows \parencite{bias_taxonomy}:
\begin{itemize}
    \item \textbf{Gender bias}: This type of bias occurs when there is an unresolvable ambiguity relating to the gender of people. Some commonly gender ambiguous words are professions such as doctor, teacher, cleaner. 
    \item \textbf{Number bias}: This bias presents itself when the English pronoun "you" has an unresolvable ambiguity concerning whether it refers to a single person ("du" in German) or multiple people ("ihr" in German). 
    \item \textbf{Formality bias}: This bias occurs when the English pronoun "you" has an unresolvable ambiguity concerning whether the subject is addressed formally ("Sie" in German) or informally ("du" in German). 
\end{itemize}

A MT system is biased if, while dealing with unresolvable ambiguities and deciding which unjustified assumptions to make, its decisions are not random. This means that the system makes certain unjustified assumptions more often than others. For example, if it consistently translates the occupation "doctor" as male, then the translator is biased \parencite{bias_taxonomy}.


% TODO: Types of languages

\parencite{Savoldi_2021}


% Available literature focuses on gender bias

%%%%%%%%%%%%%%%%%%%%%%%%%%%%%%%%%%%%%%%%%%%%%%%%%
\subsection{Bias Detection in Machine Translation}
\label{sec:Background:Bias_Detection}

% For bias detection we need:
% - Challenge sets are artificially created usually small datasets that represent some gender-related issue, such as assigning the right pronoun to a specific role
% - Automatic evaluation methods needed, because the BLEU score, normally used for assessing the quality of translations, cannot judge on the occurrence of bias [Stanovsky2019]


%%%%%%%%%%%%%%%%%%%%%%%%%%%%%%%%%%%%%%%%%%%%%%%%%
\subsection{Bias Mitigation in Machine Translation}
\label{sec:Background:Bias_Mitigation}

% Bias mitigation methods: data modification, word embeddings, architecture modification
% - Modification of the data [Costa2019]
% - Debiasing word embeddins [Zhao2018] [Bolukbasi2016]
% - Model debiasing through metadata about gender [Vanmassenhove2018]



Man is to computer programmer as woman is to
homemaker? debiasing word embeddings \parencite{bolukbasi2016man}
- Hard-debiased embeddings: post-process method for debiasing word embeddings
- Downsides: pipeline approach propagates errors; completely removes gender information from words; 
- Downsides: pipeline approach propagates errors; completely removes gender information from words;  remove valuable information in the embeddings for semantic relations between words with several meanings that are not related to the bias being treated

Men Also Like Shopping: Reducing Gender Bias Amplification using Corpus-level Constraints \parencite{Zhao_2017}
- WinoBias dataset: composed of pro-stereotype (PRO) and anti-stereotype (ANTI) subsets

Gender bias in coreference resolution: Evaluation and debiasing
methods \parencite{Zhao_2018_coreference}

Assessing gender bias in machine translation: a case study with Google Translate \parencite{Prates_2019}
- list of occupations

Getting Gender Right in Neural Machine Translation \parencite{Vanmassenhove_2018}
- develop gender-informed MT models; model debiasing through metadata
- compile a large multilingual dataset on the politics domain that contains the speaker information
- incorporating it into a MT system improves the translation quality

Neural Machine Translation Doesn't Translate Gender Coreference Right Unless You Make It \parencite{Saunders_2020_coreference}
- explore the use of word-level gender tags

Reducing Gender Bias in Neural Machine Translation as a Domain Adaptation Problem \parencite{Saunders_2020}
- propose to post-process the MT output with a lattice re-scoring module
- counterfactual data augmentation
 
Decoding and diversity in machine translation \parencite{roberts2020decoding}

GeBioToolkit: Automatic extraction of gender-balanced multilingual corpus of Wikipedia biographies \parencite{costa2019gebiotoolkit}
- fine-tuning on gender-balanced datasets based on Wikipedia biographies
- Downside: does not mitigate stereotyping harms, as it does not account for the qualitative different ways in which men and women are portrayed

On Measuring Gender Bias in Translation of Gender-neutral Pronouns \parencite{Cho_2019}

Automatically identifying gender issues in machine translation using perturbations \parencite{Gonen_2020}

"You sound just like your father": Commercial machine translation systems include stylistic biases \parencite{Hovy_2020}
- conjecture the existence of age and gender stylistic bias due to models’ under-exposure to the writings of women and younger segments of the population

Gender in danger? evaluating speech translation technology on the MuST-SHE corpus \parencite{MuST-SHE}

Fine-tuning Neural Machine Translation on Gender-Balanced Datasets \parencite{costa2020fine}


!Learning Gender-Neutral Word Embeddings \parencite{Zhao_2018_GN-GloVe}
- propose a training procedure for learning gender-neutral word embeddings
- Gender-Neutral variant of GloVe (GN-GloVe): training word embedding models with protected attributes (e.g., gender)

!Equalizing Gender Bias in Neural Machine Translation
with Word Embeddings Techniques \parencite{Escud_Font_2019}
- study the presence of gender bias in MT and give insight on
the impact of debiasing in such systems
- proposed a gender-debiased approach for NMT
- specific analysis based on correference and stereotypes to evaluate the effectiveness of our technique
- evaluate proposed system on the WMT English-Spanish benchmark task
- bilingual English-Spanish Occupations test set
- verified hypothesis that consisted on the fact that if the translation system is gender biased, the context is disregarded, while if the system is neutral, the translation is correct (since it has the information of gender in the sentence).

!Evaluating Gender Bias in Machine Translation \parencite{Stanovsky_2019}
- present the first challenge set (WinoMT) and evaluation protocol for the analysis of gender bias in machine translation (MT)
- devise an automatic gender bias evaluation method for eight target languages with grammatical gender, based on morphological analysis
- evaluate  four popular industrial MT systems and two recent state-of-the-art academic MT models (Google Translate, Microsoft Translator)
- use data introduced by two recent coreference gender-bias studies: the
Winogender \parencite{Rudinger_2018_coreference}, and the WinoBias \cite{Zhao_2018_coreference} datasets
- WinoMT: concatenating Winogender and WinoBias; equally balanced between male and female genders as well as between stereotypical and non-stereotypical gender-role assignments (e.g., a female doctor versus a female nurse)
- Measures: gender accuracy, difference in performance between male and female, difference in performance between stereotypical and non-stereotypical gender role assignments
- Fighting bias with bias: automatically creating a version of WinoMT with the adjectives “handsome” and “pretty” prepended to male and female entities, respectively -> not applicable in a real-world scenario
- Downsides: synthetic samples - controlled experiment environment, but may introduce some artificial biases; only English as source language; too small set for training easy to overfit

!Literature review: Gender Bias in Machine Translation \parencite{Savoldi_2021}
- critically review current conceptualizations of bias in light of theoretical insights from related disciplines
- summarize previous analyses aimed at assessing gender bias in MT
- discuss the mitigating strategies proposed so far
- point toward potential directions for future work


%%%%%%%%%%%%%%%%%%%%%%%%%%%%%%%%%%%%%%%%%%%%%%%%%%%%%%%%%%%%%%%%%%%%%%%%%%%%%%%%%%%%%%%%%%%%
% In Progress:
% - Unsupervised Word Sense Disambiguation (WSD) may help discover biased words
% WSD: technique in natural language processing (NLP), defined as the ability to determine which meaning of word is activated by the use of word in a particular context
% - Researching methods for Quality Estimation (QE) for detecting biases in translation
% QE: method for predicting the quality of a given translation rather than assessing how similar it is to a reference segment, E.g. multiple beams in Beam search: low confidence = high confidence for error in translation
\chapter{Related work}
\label{ch:Related_work}

Learning Gender-Neutral Word Embeddings \parencite{Zhao_2018}

Equalizing Gender Bias in Neural Machine Translation
with Word Embeddings Techniques \parencite{Escud_Font_2019}

Evaluating Gender Bias in Machine Translation \parencite{Stanovsky_2019}

Literature review: Gender Bias in Machine Translation \parencite{Savoldi_2021}


\chapter{Methodology}
\label{ch:Methodology}
% Most important chapter: talks about my contribution to the topic and what I have achieved

% Goal: Detect ambiguous words in a sentence without context
% 	- Develop method(s) to differentiate from non-ambiguous words: uncover patterns in translation and backtranslation
% 	- Finetune parameters (e.g. how often ambiguous word reoccurs in backtranslation; how many unique words in translation vs. backtranslation)
% 	(-) Suggest alternative translations to ambiguous word
%   (-) Differentiate ambiguous from biased words (cannot say if bias exists or not -> Quality Estimation)

In this chapter, we present the methods used for detecting ambiguity in MT. First, we define the problem around ambiguity that this study attempts to solve. Then, we outline the systematic approach used to solve the problem at hand.

%%%%%%%%%%%%%%%%%%%%%%%%%%%%%%%%%%%%%%%%%%%%%%%%%%%%%%%%%%%%%%%%%%%%%%%%%%%%%%%%%%%%%%%%%%%%
\section{Problem Statement}
\label{sec:Methodology:Problem}

It has been proven that NMT systems reinforce bias present in the training data (e.g., \citet{Prates_2019}, \citet{Stanovsky_2019}). This is typically the case when translating from genderless or notional languages (e.g., English) into grammatical gender languages (e.g., German). One of the most common type of bias is gender bias. 
This bias is often reflected in the way NMT systems translate occupations, since many professions are stereotyped to be either male or female dominated. For example, the occupations "doctor" would often be translated as male, while the occupation "nurse" is most commonly translated as female. 

This study focuses on gender bias, which occurs in consequence of an unresolved ambiguity, meaning that the input text does not contain information regarding the gender of the ambiguous word (e.g., "The \textit{doctor} asked for more information."). More specifically, it attempts to detect patterns in translation which could indicate the presence of an ambiguous word, which is suspected to lead to bias. For this purpose, we make use of existing NMT models based on the Transformer architecture that we described in Section \ref{sec:Background:Transformer}.
 

%%%%%%%%%%%%%%%%%%%%%%%%%%%%%%%%%%%%%%%%%%%%%%%%%%%%%%%%%%%%%%%%%%%%%%%%%%%%%%%%%%%%%%%%%%%%
\section{Approach}
\label{sec:Methodology:Approach}

% Describe the approach in an abstract way

% Input -> Translation -> Backtranslation
In this study, we take a systematic approach to discovering ambiguous words. 

\begin{enumerate}
  \item \textbf{Data Preprocessing:}
  \begin{itemize}
    \item \textbf{Sentence Extraction:} Extract sentences containing an ambiguous word, which we attempt to detect.
    \item \textbf{Replacement:} Generate a new set of sentences replacing the ambiguous word.
  \end{itemize}
  \item \textbf{Translation:} Translate both sets of sentences into the target language.
  \item \textbf{Backtranslation:} Translate the generated translations back into the original language.
  \item \textbf{Evaluation:} Generate statistical results on the translation and backtranslations.
\end{enumerate}

First, we extract sentences containing an ambiguous word, which we attempt to detect. Second, we generate a new set of sentences, replacing the ambiguous word with its disambiguated version or with a common non-ambiguous word. Then, we translate the different sets of sentences into the target language and translate the generated translations back into the original language, also called backtranslating.
Backtranslation means translating a completed translation of the input text back into the original language. The main purpose of the backtranslating technique is to be used for generating statistical results, comparing it with the original text and with its translation. On the basis of these results, we inspect the diversity of the translations and attempt to uncover recurring patterns that could prove an initial assumption. 


% Inter: between two groups
% Intra: within or inside one group
The evaluation of the results happens in two directions:
\begin{itemize}
    \item \textbf{Intra-set Evaluation:} Compare source sentences with target sentences in translation and backtranslation for each set separately.
    \item \textbf{Inter-set Evaluation:} Compare the target sentences in translation and backtranslation of the ambiguous subset with the ones of non-ambiguous subsets.
\end{itemize}


%%%%%%%%%%%%%%%%%%%%%%%%%%%%%%%%%%%%%%%%%%%%%%%%%%%%%%%%%%%%%%%%%%%%%%%%%%%%%%%%%%%%%%%%%%%%
\section{Hypothesis}
\label{sec:Methodology:Hypothesis}
The approach to detecting ambiguous words in text is based on the following hypothesis:

% Main hypothesis (intuition/hypothesis/assumption)
\begin{hyp}\label{main}
Sentences containing an ambiguity produce less diverse backtranslations than sentences without an ambiguity.
\end{hyp}

\setcounter{subhyp}{0}

%%% ??? Translation
% ??? The ambiguous word in a sentence generates more unique words in translation than the corresponding non-ambiguous word.

%%% Backtranslation
% Uniqueness evaluation: sentences
\begin{subhyp}\label{a}
Sentences containing an ambiguous word generate less unique sentences in backtranslations compared to sentences without ambiguous words.
\end{subhyp}

% Uniqueness evaluation: words
\begin{subhyp}\label{b}
Sentences containing an ambiguous word generate less unique words in backtranslations compared to sentences without ambiguous words.
\end{subhyp}

% Alignment evaluation: ambiguous word
\begin{subhyp}\label{c}
The ambiguous word in a sentence generates less unique words in backtranslation than the corresponding non-ambiguous word.
\end{subhyp}

% ??? Alignment evaluation: rest of sentence ???


Fig. \ref{fig:intuition} illustrates the intuition behind the hypothesis. There, the ambiguous word "doctor" is compared against the non-ambiguous version of "doctor", disambiguated with the prefix word "male". The word is first translated into German and the back to English. For each translation direction in the example, two unique translations are generated. In an ideal scenario, the ambiguous word generates the male and female version of the same word in translation, which leads to the same two translation options in backtranslation, while the disambiguated word generates two different words in translation, which are then translated into two other different words. In accordance with the intuition, the ambiguous word has therefore less unique words in backtranslation overall, as depicted in the example.
% propagation of diversity

\begin{figure}
  \centering
  \includegraphics[scale=0.45]{figures/intuition.png}
  \caption{Example Illustration of the Intuition}
  \label{fig:intuition}
\end{figure}

Next, we will perform a more thorough explanation of the different experimental conditions and steps followed to inspect the assumption.




\chapter{Experimental Setup}
\label{ch:Setup}

In this chapter, we outline the setup for the experiments and describe the used datasets, models, evaluation methods and tools.

%%%%%%%%%%%%%%%%%%%%%%%%%%%%%%%%%%%%%%%%%%%%%%%%%%%%%%%%%%%%%%%%%%%%%%%%%%%%%%%%%%%%%%%%%%%%
\section{Languages}
\label{sec:Setup:Languages}

% Source language: English 
% Target languages:
% - High resource: German (Germanic), French (Romance)
% - Low resource: Bulgarian (Slavic)

%%%%%%%%%%%%%%%%%%%%%%%%%%%%%%%%%%%%%%%%%%%%%%%%%
\subsection{Source Language} 
The source language used for translation is English, because translating from a notional gender language (English) into a grammatical gender language (e.g., German, French, Bulgarian) can produce biases by translating a non-gendered noun into the wrong gendered noun due to an unjustified assumption. Also, English is the most spoken language in the world, and many of the existing large datasets for training NMT models have English as either their source or target language.

%%%%%%%%%%%%%%%%%%%%%%%%%%%%%%%%%%%%%%%%%%%%%%%%%
\subsection{Target Language} 
The main target language for the base experiments of this study is German. This is primarily due to the writer's knowledge of the language, which leads to easier manual evaluation when necessary, as well as because the main occupational dataset, WinoMT \parencite{Stanovsky_2019}, also has gender evaluation for German.

% The experiments will be extended to other language families and lower-resource languages such as Bulgarian.

%%%%%%%%%%%%%%%%%%%%%%%%%%%%%%%%%%%%%%%%%%%%%%%%%%%%%%%%%%%%%%%%%%%%%%%%%%%%%%%%%%%%%%%%%%%%
\section{Datasets}
\label{sec:Setup:Datasets}

% - test dataset for general MT model performance
% - test set (challenge set or natural corpora) for assessing gender bias

We use two types of datasets — challenge sets and natural corpora. Challenge sets are synthetically created sentences, designed to be used in a controlled experiment environment to evaluate a specific phenomenon. In contrast, natural corpora are comprised of naturally occurring sentences that are meant for training and testing phenomena in real-world scenarios.

%%%%%%%%%%%%%%%%%%%%%%%%%%%%%%%%%%%%%%%%%%%%%%%%%
\subsection{Challenge Test Set}
\label{sec:Setup:Challenge_Set}

For detecting gender ambiguous words, we used the tagged challenge test set WinoMT, developed by \citet{Stanovsky_2019}. It consists of 3888 synthetic sentences presenting two human entities defined by their occupation and a subsequent pronoun that needs to be correctly resolved to match the gender of one of the entities. It also contains an equal balance between male and female gender nouns, as well as between stereotypical and non-stereotypical gender-role assignments (e.g., a female doctor versus a female nurse). We can see in Table \ref{tab:winomt} two sentences of the original dataset.

% - Downsides: synthetic samples - controlled experiment environment, but may introduce some artificial biases; only English as source language; too small set for training easy to overfit

\begin{table}
    \begin{tabularx}{\linewidth}{|X|l|l|l|}
        \hline
        \textbf{Source Sentence} & \textbf{Ambiguous word} & \textbf{Position} & \textbf{Gender} \\ \hline
        The \textbf{developer} argued with the designer because she did not like the design. & developer & 1 & female \\ \hline
        The developer argued with the \textbf{designer} because his idea cannot be implemented. & designer & 5 & male \\ \hline
    \end{tabularx}
    \caption{Example: WinoMT challenge set}
    \label{tab:winomt}
\end{table}

%%%%%%%%%%%%%%%%%%%%%%%%%%%%%%%%%%%%%%%%%%%%%%%%%
\subsection{Natural Corpora}
\label{sec:Setup:Natural_Corpora}

In order to evaluate the approach in a natural setting, we used the natural multilingual corpus MuST-SHE \parencite{MuST-SHE}, designed to evaluate the performance of NMT systems in the translation of gender for English to Spanish/French/Italian. It comprises naturally occurring instances of spoken language retrieved from MuST-C \parencite{MuST-C}, which is built on TED Talks data. The samples in the dataset are balanced between masculine and feminine phenomena. They include sentences representing four different types of gender phenomena, which are classified based
on the type of information needed to disambiguate gender translation. We are specifically interested in the fourth category, which comprises sentences, for which no gender-disambiguating information can be retrieved from context, referred previously as \textit{unresolvable ambiguity}. It contains 34 sentences in total. In Table \ref{tab:mustshe} we can see two sentences of the category. 

\begin{table} 
    \begin{tabularx}{\linewidth}{|X|l|l|}
        \hline
        \textbf{Source Sentence} & \textbf{Ambiguous Word} & \textbf{Gender} \\ \hline
        We have our cognitive biases, so that I can take a perfect history on a \textbf{patient} with chest pain. & patient & male \\ \hline
        And there was perpetual \textbf{victim} blaming when these victims came to report their crimes. & victim & female \\ \hline
    \end{tabularx}
    \caption{Example: MuST-SHE dataset}
    \label{tab:mustshe}
\end{table}

%%%%%%%%%%%%%%%%%%%%%%%%%%%%%%%%%%%%%%%%%%%%%%%%%%%%%%%%%%%%%%%%%%%%%%%%%%%%%%%%%%%%%%%%%%%%
\section{NMT Model}
\label{sec:Setup:Models}

%We use the following pre-trained NMT models:

%\subsection{English <-> German}

For the language pair English - German, we rely on the WMT19 ensemble Transformer models \footnote{https://github.com/facebookresearch/fairseq/blob/main/examples/wmt19/README.md} (En->De, De->En), developed by Facebook \parencite{WMT19} using the Fairseq sequence modeling
toolkit \parencite{fairseq}. The Workshop on Machine Translation (WMT) is the main event for machine translation and machine translation research. The WMT dataset is composed of a collection of various sources, including news commentaries and parliament proceedings. The corpus file has around 4M sentences, translated by professional translators. Facebook's WMT19 model is a state-of-the-art model, the winner in the WMT19 shared news translation task.

% TODO: statistic of the model: model size, number of layers
% ensemble of 4 models
% embedding dimension, FFN size (8192), number of heads, and number of layers


%\subsection{English <-> French}
% - French-English WMT’14 model, finetuned on MuST-C data to use for MuST-SHE evaluation

%%%%%%%%%%%%%%%%%%%%%%%%%%%%%%%%%%%%%%%%%%%%%%%%%%%%%%%%%%%%%%%%%%%%%%%%%%%%%%%%%%%%%%%%%%%%
% \section{Evaluation Methods}
% \label{sec:Setup:Evaluation}


% Evaluation procedures ought to cover both models’ general performance and gender-related issues. This is crucial to establish the capabilities and limits of mitigating strategies \cite{Savoldi_2021}.

% - Model performance metric: BLEU (translation quality)
% - Gender accuracy metric (gender bias)

%%%%%%%%%%%%%%%%%%%%%%%%%%%%%%%%%%%%%%%%%%%%%%%%%%%%%%%%%%%%%%%%%%%%%%%%%%%%%%%%%%%%%%%%%%%%
% \section{Technical Details}
% \label{sec:Setup:Technical}

%%%%%%%%%%%%%%%%%%%%%%%%%%%%%%%%%%%%%%%%%%%%%%%%%
\section{Tools}
\label{sec:Experiments:Tools}

In this section, we mention the most used tools during the development of the experiments. The programming code for the experiments is written in Python, and we use PyTorch's library for deep learning.

\paragraph{Pre-processing Tools:}
\begin{itemize}
    \item \textbf{Sacremoses \footnote{https://github.com/alvations/sacremoses}:} A pre-porcessing tool for the tokenization and detokenization of text.
    \item \textbf{Subword NMT \footnote{https://github.com/rsennrich/subword-nmt}:} A pre-processing tool for segmenting text into subword units. % used for the French models
    \item \textbf{Fast BPE \footnote{https://github.com/glample/fastBPE}:} A pre-processing tool for segmenting text into subword units. % used for the German models
    \item \textbf{Spacy \footnote{https://spacy.io/}:} An open-source library for Natural Language Processing. We use it for the detection of stop words, lemmatization and prediction of gender.
\end{itemize}

\paragraph{Translation Tools:}
\begin{itemize}
    \item \textbf{fairseq \footnote{https://github.com/facebookresearch/fairseq/tree/main}}: A sequence modeling toolkit that provides researchers and developers with tools to train custom models for translation, summarization, language modeling and other text generation tasks. It also provides reference implementations of various sequence modeling papers, including the WMT19 Transformer model \parencite{WMT19} we use for translation, as described in Section \ref{sec:Setup:Models}.

We provide reference implementations of various sequence modeling papers:
\end{itemize}

\paragraph{Word Alignment Tools:}
\begin{itemize}
    \item \textbf{Fast align \footnote{https://github.com/clab/fast\_align}:} A simple and fast unsupervised word aligner developed by \citet{fast-align}. This tool is used to align the translated sentences with the source sentences.
    \item \textbf{Awesome align \footnote{https://github.com/neulab/awesome-align}:} A contextualized embedding-based word aligner that extracts word alignments based on similarities of the tokens’ contextualized embeddings \parencite{awesome-align}. Awesome-align achieves state-of-the-art performance on five language pairs. It can extract word alignments from multilingual BERT (mBERT) and can be fine-tuned on parallel corpora for better alignment quality. We use the base mBERT model provided in the package to align the source sentences with the translated sentences.
    \item \textbf{Tercom alignment \footnote{https://github.com/jhclark/tercom}:} A tool for aligning between different translations of the source sentence. We use it to align the source sentences with the corresponding backtranslations.
\end{itemize}

%%%%%%%%%%%%%%%%%%%%%%%%%%%%%%%%%%%%%%%%%%%%%%%%%
% \subsection{Hardware Resources}
% \label{sec:Setup:Hardware}

% - GPU: GeForce GTX 1080 Ti
% - Batch size
% - Training time

\chapter{Base Experiment}
\label{ch:Base_Experiment}

This chapter describes the base experiment, which serves the purpose of applying the approach and probing the hypothesis, presented in Chapter \ref{ch:Methodology}. A summarized representation of the workflow can be seen in Fig. \ref{fig:base_workflow}. In the following, we outline the steps for executing the experiment and the results of the evaluation.

\begin{figure}
  \centering
  \includegraphics[scale=0.55]{figures/base_workflow.png}
  \caption{Base Experiment Workflow}
  \label{fig:base_workflow}
\end{figure}

%%%%%%%%%%%%%%%%%%%%%%%%%%%%%%%%%%%%%%%%%%%%%%%%%%%%%%%%%%%%%%%%%%%%%%%%%%%%%%%%%%%%%%%%%%%%
\section{Data Pre-processing}
\label{sec:Base_Experiment:Pre-processing}
The first step in conducting the base experiment is preprocessing the dataset. We use the artificially created WinoMT challenge set, presented in Subsection \ref{sec:Setup:Challenge_Set}. The sentences in this dataset usually consist of two gender ambiguous occupations and a context, containing disambiguation information about one of the occupations. We take the following steps to preprocess the sentences:

\begin{enumerate}
  \item \textbf{Sentence Extraction:}  
  In order to obtain fully ambiguous sentences, we remove the context information from the sentences and obtain a subset of 335 sentences from the type: "The developer argued with the designer.".
  To remove additional detection overhead, we want to have a single ambiguous word per sentence. For this purpose, we replace the second ambiguous word with a non-ambiguous proper noun, e.g. "John". 
  \item \textbf{Replacement:} 
  As next, we generate a new set of sentences, replacing the ambiguous word with different techniques:
  \begin{itemize}
      \item \textbf{Disambiguation:} We use the gender-defining adjectives \textit{male} and \textit{female} in front of the gender-ambiguous word. This technique is meant to force the translator to make the right decision regarding gender. % gender forcing
      \item \textbf{Common Words:} We replace the ambiguous word with the following common gender non-ambiguous words: \textit{man, woman, girl, guy, boy}. This method serves as a baseline for comparison against the disambiguated occupations.
  \end{itemize}
\end{enumerate}

Table \ref{tab:preprocessing} shows the generated subsets obtained by disambiguating the base ambiguous sentence "The developer argued with John.".

\begin{table}
    \begin{tabularx}{\linewidth}{|l|X|l|}
        \hline
        \textbf{Replacement Method} & \textbf{Source Sentence} & \textbf{Source Word} \\ \hline
        \multirow{2}{*}{Disambiguation} & The \textbf{male developer} argued with John. & developer \\
        & The \textbf{female developer} argued with John. & developer \\ \hline
        \multirow{5}{*}{Common Words} & The \textbf{man} argued with John. & man \\
        & The \textbf{woman} argued with John. & woman \\
        & The \textbf{girl} argued with John. & girl \\
        & The \textbf{guy} argued with John. & guy \\
        & The \textbf{boy} argued with John. & boy \\ \hline
    \end{tabularx}
    \caption{Example: Disambiguation subsets for the baseline sentence "The developer argued with John."}
    \label{tab:preprocessing}
\end{table}

%%%%%%%%%%%%%%%%%%%%%%%%%%%%%%%%%%%%%%%%%%%%%%%%%%%%%%%%%%%%%%%%%%%%%%%%%%%%%%%%%%%%%%%%%%%%
\section{Translation}
\label{sec:Base_Experiment:Translation}

The next step in conducting the experiments is translating the subsets of sentences. This is executed in two steps:

\begin{enumerate}
    \item \textbf{Translation Source -> Target:} 
    First, the subsets are translated in the target language.
    \item \textbf{Backtranslation Target -> Source:}
    The second step involves translating the translations back into the source language.
\end{enumerate}


% TODO: Decoding/search algorithm/strategy, nbest size (define what nbest means)
% We use two different decoding algorithms to compare the results: Beam search and Sampling. 
% In each step we generate nbest lists of different sizes: 10 and 100.

% ! \citet{roberts2020decoding} prove that beam search unlike sampling is skewed toward the generation of more frequent (masculine) pronouns, as it leads models to an extreme operating point that exhibits zero variability.

%%%%%%%%%%%%%%%%%%%%%%%%%%%%%%%%%%%%%%%%%%%%%%%%%%%%%%%%%%%%%%%%%%%%%%%%%%%%%%%%%%%%%%%%%%%%
\section{Word alignment}
\label{sec:Base_Experiment:Alignment}

% ??? maybe move this to Methodology, because this is the same for all experiments

In order to assign the words in the source sentence to their counterparts in the translations, we use two different alignment methods:

\begin{enumerate}
    \item Source-to-translation (\textit{fast\_align}, \textit{awesome-align}): This alignment method aligns from the source language to the target language.
    \item Translation-to-translation (\textit{Tercom}): This alignment method aligns between two translations.
\end{enumerate}

We use the first method to map each word in the source sentence to its translations and backtranslations in the target nbest lists. We do this in a two-step way, depicted in Fig. \ref{fig:alignment}. First, we align between the source sentence and the sentences in the nbest translations and extract the translations for each word. Then, we align between the translations and the backtranslations and extract the corresponding backtranslations resulting from the aligned translations of each word. 

We use the results from the second method as a baseline for comparison with the first method and to detect possible errors, which may occur in the information transfer between the two steps in the first method.

\begin{figure}
  \centering
  \includegraphics[scale=0.5]{figures/alignment.png}
  \caption{Example Illustration: 2-step mapping from source to translation and backtranslation}
  \label{fig:alignment}
\end{figure}

%%%%%%%%%%%%%%%%%%%%%%%%%%%%%%%%%%%%%%%%%%%%%%%%%%%%%%%%%%%%%%%%%%%%%%%%%%%%%%%%%%%%%%%%%%%%
\section{Evaluation}
\label{sec:Base_Experiment:Evaluation}

The last step in the experiments involves evaluating the translations and backtranslations to detect patterns using different statistical methods. These methods aim to probe the initial Hypothesis \ref{main}, discussed in Section \ref{sec:Methodology:Approach}. We apply all methods to all subsets and extract diversity information regarding the subsets themselves, as well as compare the results of the subsets against each other.

% TODO: define formally, maybe as a formula, pseudocode ???

%%%%%%%%%%%%%%%%%%%%%%%%%%%%%%%%%%%%%%%%%%%%%%%%%
\subsection{Recurrence Evaluation}
\label{sec:Base_Experiment:Statistics:Recurrence}
We evaluate how many of the source sentences and source words reoccur in the backtranslations:

\begin{enumerate}
    \item[1. ] Gather the backtranslations for each source sentences.
    \item[2a. ] Count the number of source sentences that reappear in their list of backtranslations.
    \item[2b. ] Count the number of source sentences which contain the source word in their list of backtranslations.
\end{enumerate}

The purpose of this evaluation is to determine which of the subsets are able to reconstruct more of the original source sentences/words.

%%%%%%%%%%%%%%%%%%%%%%%%%%%%%%%%%%%%%%%%%%%%%%%%%
\subsection{Uniqueness Evaluation}
\label{sec:Base_Experiment:Statistics:Uniqueness}
We evaluate the number of unique words and sentences in the translations and backtranslations for each sentence of the subsets. 

For the sake of the evaluation, we follow this routine (\textit{[translations | backtranslations]} denotes that we follow the same step for both the translations and backtranslations):
\begin{enumerate}
    \item[1. ] Collect the \textit{[translations | backtranslations]} for each source sentences.
    % sentence level
    \item[2a. ] Count how many of the \textit{[translations | backtranslations]} are unique. 
    % word level
    \item[2b. ] Count how many unique words there are in the \textit{[translations | backtranslations]} and normalize the number by the total amount of words. 
    \item[3. ] Average the result for all sentences.
\end{enumerate}

We use this method to probe the Hypotheses \ref{a} and \ref{b}. 

%%%%%%%%%%%%%%%%%%%%%%%%%%%%%%%%%%%%%%%%%%%%%%%%%
\subsection{Gender Evaluation}
\label{sec:Base_Experiment:Statistics:Gender}
We perform the evaluation of gender on the translations of the subsets.  First, we align each word in the source sentence with its corresponding word in the translations and backtranslations (see Subsection \ref{sec:Base_Experiment:Alignment} for more detail). Then, we perform the following steps:

\begin{enumerate}
    \item[1. ] Gather the translations of the source word for each source sentence.
    \item[2. ] Detect the gender of the translations for each source sentence.
    \item[3a. ] Determine the proportion of source sentences producing \textit{male} versus \textit{female}.
    \item[3b. ] Calculate how many of the source sentences produce \textit{both genders}. 
\end{enumerate}

The purpose of this method is to assess if the translations produce the right gender (in the non-ambiguous subsets) or both genders (in the ambiguous subset) and how often they produce both genders for each subset.

%%%%%%%%%%%%%%%%%%%%%%%%%%%%%%%%%%%%%%%%%%%%%%%%%
\subsection{Alignment Evaluation}
\label{sec:Base_Experiment:Statistics:Alignment}
Another form of evaluation is based on the alignment of the words between the source sentence, the translations and backtranslations (see Subsection \ref{sec:Base_Experiment:Alignment} for more detail).

In order to assess the translations and backtranslations of the \textbf{source word}, we do:
\begin{enumerate}
    \item[1. ] Collect all \textit{[translations | backtranslations]} of the source word.
    \item[2. ] Count how many of the \textit{[translations | backtranslations]} are unique.
    \item[3. ] Average the result for all sentences.
\end{enumerate}

Similarly, to assess the translations and backtranslations of the \textbf{rest of the sentence} excluding the source word, we do:
\begin{enumerate}
    \item[1. ] Collect all \textit{[translations | backtranslations]} of the sentence rest.
    \item[2. ] Count how many of the \textit{[translations | backtranslations]} are unique.
    \item[3. ] Average the result for all sentences.
\end{enumerate}

The idea behind this evaluation method is to assess the Hypothesis \ref{c}.

Next, we will present the results of these statistical evaluations of the subsets.

%%%%%%%%%%%%%%%%%%%%%%%%%%%%%%%%%%%%%%%%%%%%%%%%%%%%%%%%%%%%%%%%%%%%%%%%%%%%%%%%%%%%%%%%%%%%
\section{Results}
\label{ch:Base_Experiment:Results}

% - Translation quality
% - Gender bias quality

% Finding patterns in statistical results
% e.g. influence of language, context

% BLEU score on WinoMT: not possible, because no reference translations

% Variables to consider:
% - Word alignment method
% - Disambiguation method 
% ?- Search method
% - Nbest size


%%%%%%%%%%%%%%%%%%%%%%%%%%%%%%%%%%%%%%%%%%%%%%%%%
\subsection{Recurrence Evaluation Results}
\label{ch:Base_Experiment:Results:Recurrence}

The results from the evaluation of recurrence for beam search with beam size 10 are listed in Table \ref{tab:recurrence_10}.
% Highest score
As we can observe, the average from the subsets of common words presents the highest score in both the recurring sentences and words. This is to be expected, because the words in these subsets are most generic and have the highest probability of being predicted, compared to the occupational words from the WinoMT sentences in the other three subsets. 

% Interesting findings
Most interestingly, the female-disambiaguated subset has the lowest score for occurring sentences. When investigating the results, we found some discrepancy between the way "female" and "male" are translated. The "female" prefix is very often lost in the backtranslation, which results in the backtranslated sentence being regarded as differing from the source sentence. In contrast, the "male" prefix is most often preserved, resulting in the same sentence in backtranslation. We illustrate this with the following examples:

\begin{itemize}
    \item \textbf{Source (EN):} The \textit{female} developer argued with John. \\
    \textbf{Translation (DE):} Die Entwicklerin argumentierte mit John. \\
    \textbf{Backtranslation (EN):} The developer argued with John.
    
    \item \textbf{Source (EN):} The \textit{male} developer argued with John. \\
    \textbf{Translation (DE):} Der \textit{männliche} Entwickler argumentierte mit John. \\
    \textbf{Backtranslation (EN):} The \textit{male} developer argued with John.
\end{itemize}

As we can see, the "male" prefix is translated to its corresponding word in German "männliche", while the "female" prefix is lost in the translation, but its meaning is reflected in the female gender of the German word for developer "Entwicklerin".

Also, both disambiguation subsets sometimes generate the opposite gender with the correct prefix, for example, "der \textit{weibliche} Entwickler" (the \textit{female} male developer) and "die \textit{männliche} Entwicklerin" (the \textit{male} female developer). This presents a contradiction that influences the . 

We note that the findings from these results are important and will have an effect on the further experiments.

Table \ref{tab:recurrence_100} shows the results from the evaluation of recurrence for beam search with beam size 100. Here we can observe that more female sentences are reoccurring in backtranslation compared to beam 100, which means that increasing the beam increases the possibility for preservation of the "female" prefix.
Also, with beam size 100 the source word reappears for all source sentences in the backtranslations.

\begin{table} 
    \begin{tabularx}{\linewidth}{|X|XXXX|}
        \hline
         & \textbf{Ambiguous} & \textbf{Disambiguated (male)} & \textbf{Disambiguated (female)} & \textbf{Non-ambiguous average} \\ \hline
         \textbf{Sentences} & 295/335 & 293/335 & 118/335 & \underline{308/335} \\ 
         \textbf{Words} & 329/335 & 330/335 & 314/335 & \underline{335/335} \\ \hline
    \end{tabularx}
    \caption{\textbf{Recurrence Evaluation Results: Beam Size 10}. English-German. Backtranslation. Beam search with beam size 10. Nbest size 10. Highest scores are underlined. \\ First row: number of source sentences that reappear in the backtranslations. \\ Second row: number of source sentences which contain the source word in the backtranslations.}
    \label{tab:recurrence_10}
\end{table}

\begin{table} 
    \begin{tabularx}{\linewidth}{|X|XXXX|}
        \hline
         & \textbf{Ambiguous} & \textbf{Disambiguated (male)} & \textbf{Disambiguated (female)} & \textbf{Non-ambiguous average} \\ \hline
         \textbf{Sentences} & 329/335 & \underline{330/335} & 281/335 & 329/335 \\ 
         \textbf{Words} & 335/335 & 335/335 & 335/335 & 335/335 \\ \hline
    \end{tabularx}
    \caption{\textbf{Recurrence Evaluation Results: Beam Size 100}. English-German. Backtranslation. Beam search with beam size 100. Nbest size 100. Highest scores are underlined. \\ First row: number of source sentences that reappear in the backtranslations. \\ Second row: number of source sentences which contain the source word in the backtranslations.}
    \label{tab:recurrence_100}
\end{table}

%%%%%%%%%%%%%%%%%%%%%%%%%%%%%%%%%%%%%%%%%%%%%%%%%
\subsection{Uniqueness Evaluation Results}
\label{ch:Base_Experiment:Results:Uniqueness}

The results from the evaluation of uniqueness are listed in Table \ref{tab:uniqueness_translation} for translation and Table \ref{tab:uniqueness_backtranslation} for backtranslation.
Since we use the beam search algorithm for decoding with beam size 10 and nbest size 10, we expect it to generate 10 unique sentences per translation, which is almost always the case, as we see from the score for the number of unique sentences in translations. 

% Interesting findings
Most notable are the results for the number of unique backtranslations. As we can see, the ambiguous subset produces the least amount of unique sentences in backtranslation, which proves Hyp. \ref{a}. In Table \ref{tab:uniqueness_backtranslation_100} we can see the results for number of unique backtranslations with beam size 100, which confirm this result. However, when regarding the results for the different common words separately instead of averaged, as we can see in Fig. \ref{fig:range}, the minimum value in the range, 43.9 (for the word "boy") for beam size 10 and 3074.01 (for the word "guy") for beam size 100, is still lower for the common words compared to the ambiguous subset, 45.98 for beam size 10 and 3181.51 for beam size 100.

The results for the number of unique words in translation and backtranslation are inconclusive. Considering Hyp. \ref{b}, we expected the ambiguous subset to generate the least unique words in backtranslation, but this is not the case.

\begin{table} 
    \begin{tabularx}{\linewidth}{|X|XXXX|}
        \hline
         & \textbf{Ambiguous} & \textbf{Disambiguated (male)} & \textbf{Disambiguated (female)} & \textbf{Non-ambiguous average} \\ \hline
         \textbf{Sentences} & 9.94/10 & 9.95/10 & 9.87/10 & \underline{9.97/10} \\ 
         \textbf{Words} & \underline{0.205} & 0.19 & 0.201 & 0.202 \\ \hline
    \end{tabularx}
    \caption{\textbf{Uniqueness Evaluation Results for Translation}. English-German. Beam search with beam size 10. Nbest size 10. Highest scores are underlined. \\ First row: Averaged number of unique sentences per source sentence out of 10 translations. \\ Second row: Averaged number of unique words per source sentence, normalized by the average total number of words in 10 translations.}
    \label{tab:uniqueness_translation}
\end{table}

\begin{table} 
    \begin{tabularx}{\linewidth}{|X|XXXX|}
        \hline
         & \textbf{Ambiguous} & \textbf{Disambiguated (male)} & \textbf{Disambiguated (female)} & \textbf{Non-ambiguous average} \\ \hline
         \textbf{Sentences} & 45.98/100 & 50.73/100 & \underline{50.82/100} & 47.06/100 \\ 
         \textbf{Words} & 0.044 & 0.039 & 0.043 & \underline{0.045} \\ \hline
    \end{tabularx}
    \caption{\textbf{Uniqueness Evaluation Results for Backtranslation: Beam Size 10}. English-German. Beam search with beam size 10. Nbest size 10. Best results are underlined. \\ First row: Averaged number of unique sentences per source sentence out of 10 translations. \\ Second row: Averaged number of unique words per source sentence, normalized by the average total number of words in 100 backtranslations.}    \label{tab:uniqueness_backtranslation}
\end{table}

\begin{table} 
    \begin{tabularx}{\linewidth}{|X|XXXX|}
        \hline
         & \textbf{Ambiguous} & \textbf{Disambiguated (male)} & \textbf{Disambiguated (female)} & \textbf{Non-ambiguous average} \\ \hline
         \textbf{Sentences} & 3181.51/10000 & 3391.55/10000 & \underline{3424.98/10000} & 3297.48/10000 \\ \hline
    \end{tabularx}
    \caption{\textbf{Uniqueness Evaluation Results for Backtranslation: Beam Size 100}. English-German. Beam search with beam size 100. Nbest size 100. Best results are underlined. \\ First row: Averaged number of unique sentences per source sentence out of 100 translations. \\ Second row: Averaged number of unique words per source sentence, normalized by the average total number of words in 10000 backtranslations.}
    \label{tab:uniqueness_backtranslation_100}
\end{table}

%%% Range of uniqueness
\begin{figure}
     \centering
     
     \begin{subfigure}{0.49\textwidth}
         \centering
         \includegraphics[width=\textwidth]{figures/uniqueness/range_beam_10.png}
         \caption{Beam 10}
         \label{fig:three sin x}
     \end{subfigure}
     \hfill
     \begin{subfigure}{0.49\textwidth}
         \centering
         \includegraphics[width=\textwidth]{figures/uniqueness/range_beam_100.png}
         \caption{Beam 100}
         \label{fig:five over x}
     \end{subfigure}
     
    \caption{Comparison Between the Number of Unique Backtranslations for Common Words and Ambiguous Subsets}
    \label{fig:range}

\end{figure}

%%%%%%%%%%%%%%%%%%%%%%%%%%%%%%%%%%%%%%%%%%%%%%%%%
\subsection{Gender Evaluation Results}
\label{ch:Base_Experiment:Results:Gender}

The results from the evaluation of gender are listed in Table \ref{tab:gender_percent_10} for beam size 10 and in Table \ref{tab:gender_percent_100} for beam size 100. We can observe that, as expected, the subset of disambiguated with "male" sentences has predominantly male translations, and similarly the subset of disambiguated with "female" sentences has mostly female translations. The same applies to the male words "man", "guy" and "boy", as well as the female word "woman". The female word "girl" presents an exception, because in German it is a neutral noun.

Also, as expected, the ambiguous source sentences produce the most translations of both genders, while the common non-ambiguous words produce the least. Despite this, the disambiguation subsets still have a rather high amount of sentences producing both genders. 

Interestingly, when comparing the disambiguation subsets, the disambiguation with "female" seems to be more successful overall, with more sentences producing the right gender and less of both genders appearing in the translations.

Fig. \ref{fig:gender_pie_10} further shows the influence of increasing the beam size tenfold. We can observe that more of both genders occur in translation with beam 100 compared to beam 10. Also, there is more balance between female and male in the translations of the ambiguous subset, with the difference between 78.66\% and 14.88\% being smaller than between 86.27\% and 12.81\%. But on the other hand, more male gender translations occur in the female-disambiguated subset, which is a downside.

\begin{table} 
    \begin{tabularx}{\linewidth}{|X|XXXX|}
        \hline
         & \textbf{Ambiguous} & \textbf{Disambiguated (male)} & \textbf{Disambiguated (female)} & \textbf{Non-ambiguous} \\ \hline
         \textbf{Male} & 86.27\% & 89.46\% & 6.81\% & \textit{man}: 95.01\% \\
         &&&& \textit{woman}: 0.51\% \\
         &&&& \textit{girl}: 0.39\% \\
         &&&& \textit{guy}: 93.07\% \\
         &&&& \textit{boy}: \underline{96.15\%} \\ \hline
         \textbf{Female} & 12.81\% & 11.19\% & 92.33\% & \textit{man}: 0.18\% \\ 
         &&&& \textit{woman}: \underline{96.69\%} \\
         &&&& \textit{girl}: 0.81\% \\
         &&&& \textit{guy}: 0.18\% \\
         &&&& \textit{boy}: 0.27\% \\\hline
         \textbf{Both genders} & \underline{38.21\%} & 35.22\% & 28.06\% & average: 0.72\% \\ \hline
    \end{tabularx}
    \caption{\textbf{Gender Evaluation Results: Beam Size 10}. English-German. Translation. Beam search with beam size 10. Nbest size 10. Highest scores are underlined. \\ First and second row: Percentage of the source sentences producing male versus female translations. \\ Third row: Percentage of the source sentences producing both genders in translation.}
    \label{tab:gender_percent_10}
\end{table}

\begin{table} 
    \begin{tabularx}{\linewidth}{|X|XXXX|}
        \hline
         & \textbf{Ambiguous} & \textbf{Disambiguated (male)} & \textbf{Disambiguated (female)} & \textbf{Non-ambiguous} \\ \hline
         \textbf{Male} & 78.66\% & 81.96\% & 8.27\% & \textit{man}: 83.80\% \\
         &&&& \textit{woman}: 0.59\% \\
         &&&& \textit{girl}: 0.75\% \\
         &&&& \textit{guy}: 78.59\% \\
         &&&& \textit{boy}: \underline{86.48\%} \\ \hline
         \textbf{Female} & 14.88\% & 14.24\% & 86.90\% & \textit{man}: 0.46\% \\ 
         &&&& \textit{woman}: \underline{88.57\%} \\
         &&&& \textit{girl}: 5.66\% \\
         &&&& \textit{guy}: 0.98\% \\
         &&&& \textit{boy}: 0.75\% \\\hline
         \textbf{Both genders} & \underline{91.64\%} & 91.34\% & 82.98\% & average: 23.04\% \\ \hline
    \end{tabularx}
    \caption{\textbf{Gender Evaluation Results: Beam Size 100}. English-German. Translation. Beam search with beam size 100. Nbest size 100. Highest scores are underlined. \\ First and second row: Percentage of the source sentences producing male versus female translations. \\ Third row: Percentage of the source sentences producing both genders in translation.}
    \label{tab:gender_percent_100}
\end{table}

%%% Pie charts
\begin{figure}
     \centering
     
     \begin{subfigure}{\textwidth}
         \centering
         \includegraphics[width=\textwidth]{figures/gender/beam_10.png}
         \caption{Beam 10}
         \label{fig:three sin x}
     \end{subfigure}
     
     \begin{subfigure}{\textwidth}
         \centering
         \includegraphics[width=\textwidth]{figures/gender/beam_100.png}
         \caption{Beam 100}
         \label{fig:five over x}
     \end{subfigure}
     
     
    \caption{Gender Representation in Translation}
    \label{fig:gender_pie_10}
\end{figure}

%%%%%%%%%%%%%%%%%%%%%%%%%%%%%%%%%%%%%%%%%%%%%%%%%
\subsection{Alignment Evaluation Results}
\label{ch:Base_Experiment:Results:Alignment}

The results from the evaluation of gender for beam 10 and 100 are listed in Tables \ref{tab:alignment_translation_10} and \ref{tab:alignment_translation_100} for translation and Tables \ref{tab:alignment_backtranslation_10} and \ref{tab:alignment_backtranslation_100} for backtranslation.

% ??? - The rest of the sentence excluding the ambiguous word should have more unique words than the rest of the sentence excluding the disambigauted word
% ??? - The ambiguous word in a sentence generates more unique words in backtranslation than the rest of the sentence.

% Translation
The alignment results for translation feature unique words with and without gender information. For example, the male and female German word for developer: "Entwickler" and "Entwicklerin", are considered one unique word when disregarding gender information. The removal of gender information is performed using a rule-based approach. Despite this, the assumption that this would reduce the number of unique words for the ambiguous subset compared to the other subsets, is not entirely confirmed. This is only the case when we compare the female-disambiguated subset with the ambiguous subset.


% Difference between unique words per ambiguous words vs. rest of sentence for original and unambiguous words average (2.38 – 1.87 vs 1.70 – 1.94)
A notable trend in the results is that the non-ambiguous subset has the significantly lowest score in uniqueness for the translations of the source word, but it also has the highest score for the rest of the sentence. But also, the scores for the source word and the rest of the sentence are closer together for the non-ambiguous subset than the scores for the ambiguous subset. This is also an indication of the stronger tendency of the decoding algorithm to put more emphasis on the ambiguous word rather than the rest of the sentence, when such an ambiguous word is present.

The results for beam size 100 also show a similar trend.

% Backtranslation
While the results from the translation are not conclusive, the results from the backtranslation exhibit a noticeable pattern. We can see that for the source word, the non-ambiguous subset has the least amount of unique backtranslations, contrary to the expectation from Hyp. \ref{c}, which postulated that the ambiguous subset should produce the least unique backtranslations. However, the ambiguous subset has less unique translations than the female-disambiguated subset, which partially confirms the hypothesis. The same results are observed with beam size 100.

% Interesting results
Interestingly, for beam size 10, both in translation and backtranslation the rest of the sentence for the non-ambiguous subset produces the most diverse translations compared to the ambiguous subset. This may indicate that more emphasis is given on diversity of the rest of the sentence, when the source word is unambiguous itself. However, the same observation cannot be made for beam size 100, where the rest of the sentence for the ambiguous subset produces the most diverse translations.

% TODO: Difference between FA and AA

\begin{table} 
    \begin{tabularx}{\linewidth}{|X|XXXX|}
        \hline
         & \textbf{Ambiguous} & \textbf{Disambiguated (male)} & \textbf{Disambiguated (female)} & \textbf{Non-ambiguous average} \\ \hline
         \textbf{Source word (FA, WIG)} & 2.87/10 & 2.83/10 & \underline{3.02/10} & 1.83/10 \\
         \textbf{Source word (FA, WOG)} & 2.66/10 & 2.64/10 & \underline{2.81/10} & 1.83/10 \\
         \textbf{Source word (AA, WIG)} & 2.38/10 & \underline{2.43/10} & 2.33/10 & 1.70/10 \\ 
         \textbf{Source word (AA, WOG)} & 2.66/10 & 2.64/10 & \underline{2.81/10} & 1.83/10 \\\hline 
         \textbf{Sentence rest (AA)} & 1.87/10 & 1.67/10 & 1.64/10 & \underline{1.94/10} \\ \hline
    \end{tabularx}
    \caption{\textbf{Alignment Evaluation Results for Translation: Beam Size 10}. English-German. Beam search with beam size 10. Nbest size 10. Highest scores are underlined. FA: \textit{fast\_align}, AA: \textit{awesome-align}. WIG (with gender): gender information preserved, WOG (without gender): gender information removed. \\ First and second row: Averaged number of unique translations of the source word per source sentence in the 10 translations. \\ Third row: Averaged number of unique translations of the sentence rest per source sentence in the 10 translations.}
    \label{tab:alignment_translation_10}
\end{table}

\begin{table} 
    \begin{tabularx}{\linewidth}{|X|XXXX|}
        \hline
         & \textbf{Ambiguous} & \textbf{Disambiguated (male)} & \textbf{Disambiguated (female)} & \textbf{Non-ambiguous average} \\ \hline
         \textbf{Source word (FA, WIG)} & 12.37/100 & 13.70/100 & \underline{14.76/100} & 7.18/100 \\
         \textbf{Source word (FA, WOG)} & 11.04/100 & 12.47/100 & \underline{13.38/100} & 7.18/100 \\
         \textbf{Source word (AA, WIG)} & 10.81/100 & 12.12/100 & \underline{13.12/100} & 5.26/100 \\ 
         \textbf{Source word (AA, WOG)} & 9.46/100 & 10.88/100 & \underline{11.75/100} & 5.26/100 \\\hline 
         \textbf{Sentence rest (AA)} & 6.52/100 & 5.39/100 & 5.85/100 & \underline{7.23/100} \\ \hline
    \end{tabularx}
    \caption{\textbf{Alignment Evaluation Results for Translation: Beam Size 100}. English-German. Beam search with beam size 100. Nbest size 100. Highest scores are underlined. FA: \textit{fast\_align}, AA: \textit{awesome-align}. WIG (with gender): gender information preserved, WOG (without gender): gender information removed. \\ First and second row: Averaged number of unique translations of the source word per source sentence in the 100 translations. \\ Third row: Averaged number of unique translations of the sentence rest per source sentence in the 100 translations.}
    \label{tab:alignment_translation_100}
\end{table}

\begin{table} 
    \begin{tabularx}{\linewidth}{|X|XXXX|}
        \hline
         & \textbf{Ambiguous} & \textbf{Disambiguated (male)} & \textbf{Disambiguated (female)} & \textbf{Non-ambiguous average} \\ \hline
         \textbf{Source word (FA)} & 8.84/100 & 8.32/100 & \underline{9.79/100} & 5.25/100 \\ 
         \textbf{Source word (AA)} & 7.48/100 & 7.04/100 & \underline{7.85/100} & 4.80/100 \\ 
         \textbf{Source word (Tercom)} & 8.02/100 & 7.90/100 & \underline{9.82/100} & 6.52/100 \\ \hline
         \textbf{Sentence rest (AA)} & 4.13/100 & 3.72/100 & 3.73/100 & \underline{4.66/100} \\ \hline
    \end{tabularx}
    \caption{\textbf{Alignment Evaluation Results for Backtranslation}. English-German. Beam search with beam size 10. Nbest size 10. Highest scores are underlined. FA: \textit{fast\_align}, AA: \textit{awesome-align}. \\ First-third row: Averaged number of unique backtranslations of the source word per source sentence in the 100 backtranslations. \\ Fourth row: Averaged number of unique backtranslations of the sentence rest per source sentence in the 100 backtranslations.}
    \label{tab:alignment_backtranslation_10}
\end{table}

\begin{table} 
    \begin{tabularx}{\linewidth}{|X|XXXX|}
        \hline
         & \textbf{Ambiguous} & \textbf{Disambiguated (male)} & \textbf{Disambiguated (female)} & \textbf{Non-ambiguous average} \\ \hline
         \textbf{Source word (FA)} & 147.93/10000 & 148.34/10000 & \underline{163.53/10000} & 69.76/10000 \\ 
         \textbf{Source word (AA)} & 120.08/10000 & 121.96/10000 & \underline{136.14/10000} & 48.82/10000 \\ \hline 
         \textbf{Sentence rest (AA)} & \underline{68.00/10000} & 66.39/10000 & 63.41/10000 & 66.06/10000 \\ \hline
    \end{tabularx}
    \caption{\textbf{Alignment Evaluation Results for Backtranslation}. English-German. Beam search with beam size 100. Nbest size 100. Highest scores are underlined. FA: \textit{fast\_align}, AA: \textit{awesome-align}. \\ First-third row: Averaged number of unique backtranslations of the source word per source sentence in the 10000 backtranslations. \\ Fourth row: Averaged number of unique backtranslations of the sentence rest per source sentence in the 10000 backtranslations.}
    \label{tab:alignment_backtranslation_100}
\end{table}




% Test

Test fig. \ref{fig:three graphs}

\begin{figure}
     \centering
     
     \begin{subfigure}{0.49\textwidth}
         \centering
         \includegraphics[width=\textwidth]{figures/unique_back_original.png}
         \caption{$y=3\sin x$}
         \label{fig:three sin x}
     \end{subfigure}
     \hfill
     \begin{subfigure}{0.49\textwidth}
         \centering
         \includegraphics[width=\textwidth]{figures/unique_back_original.png}
         \caption{$y=5/x$}
         \label{fig:five over x}
     \end{subfigure}
     \begin{subfigure}{0.49\textwidth}
         \centering
         \includegraphics[width=\textwidth]{figures/unique_back_original.png}
         \caption{$y=3\sin x$}
         \label{fig:three sin x}
     \end{subfigure}
     \hfill
     \begin{subfigure}{0.49\textwidth}
         \centering
         \includegraphics[width=\textwidth]{figures/unique_back_original.png}
         \caption{$y=5/x$}
         \label{fig:five over x}
     \end{subfigure}
        \caption{Three simple graphs}
        \label{fig:three graphs}

\end{figure}
\input{sections/base_results.tex}
\chapter{Real-world Experiment}
\label{ch:Real_Experiment} 

This chapter describes an experiment in a real-world setting. The purpose of this experiment is to test the hypothesis in natural conditions. In the following, we outline the steps for executing the
experiment and the results from the evaluation.

%%%%%%%%%%%%%%%%%%%%%%%%%%%%%%%%%%%%%%%%%%%%%%%%%%%%%%%%%%%%%%%%%%%%%%%%%%%%%%%%%%%%%%%%%%%%
\section{Data Extraction}
\label{sec:Real_Experiment:Extraction}

First, we extract the sentences from the MuST-SHE dataset, as presented in Subsection \ref{sec:Setup:Natural_Corpora}. We choose 10 sentences for the experiment, that contain no context information regarding the gender of the ambiguous words. The sentence set is balanced, containing 5 sentences of each two genders - male and female. All sentences are listed in Table \ref{tab:mustshe_sentences}.

\begin{table} 
    \begin{tabularx}{\linewidth}{|X|l|l|}
        \hline
        \textbf{Source Sentence} & \textbf{Ambiguous Word(s)} & \textbf{Gender} \\ \hline
        So now Thomson becomes the more likely \textbf{suspect}. & suspect & male \\ \hline
        There was one black \textbf{professor} and one black assistant \textbf{dean}. & professor, dean & male \\ \hline
        We have our cognitive biases, so that I can take a perfect history on a \textbf{patient} with chest pain. & patient & male \\ \hline
        That's the \textbf{officer} who emailed me back, saying I think you can have a few classes with us.  & officer & male \\ \hline
        Steve, a physician, told me about a \textbf{doctor} that he worked with who was never very respectful, especially to junior staff and nurses. & doctor & male \\ \hline
        What do you think a batting average for a \textbf{cardiac surgeon} or a \textbf{nurse practitioner} or an \textbf{orthopedic surgeon}, an \textbf{OBGYN}, a \textbf{paramedic} is supposed to be? & \makecell[l]{surgeon, \\ nurse practitioner, \\ OBGYN, paramedic} & female \\ \hline
        Fortunately for Mama Jane and her \textbf{friend}, a \textbf{donor} had provided treatment so that we could take them to the nearest hospital three hours away. & friend, donor & female \\ \hline
        The three words are: Do you remember? "Do you remember that patient you sent home?" the other \textbf{nurse} asked matter-of-factly. "Well she's back," in just that tone of voice. & nurse & female \\ \hline
        This one comes from a note that a \textbf{student} sent me after I gave a lecture about arousal nonconcordance. & student & female \\ \hline
        At the end of a conference in a hotel lobby once, I'm literally on my way out the door and a \textbf{colleague} chases me down. "Emily, I just have a really quick question."  & colleague & female \\ \hline
    \end{tabularx}
    \caption{\textbf{MuST-SHE Extracted Sentences.} 4th Category: No gender-disambiguating information can be retrieved. 10 sentences in total.}
    \label{tab:mustshe_sentences}
\end{table}


%%%%%%%%%%%%%%%%%%%%%%%%%%%%%%%%%%%%%%%%%%%%%%%%%%%%%%%%%%%%%%%%%%%%%%%%%%%%%%%%%%%%%%%%%%%%
\section{Data Preprocessing}
\label{sec:Real_Experiment:Preprocessing}

As next, we preprocess the data by unmasking for each word in each original sentence. For unmasking we use the BERT base model, introduced in Section \ref{sec:Experiments:Tools}. The model generates five most probable words for each masked word. We replace the unmasked words in the sentences with the generated words. For each original sentence, we have five times the number of words in the sentence unmasked sentences.

\paragraph{Sets of Sentences} We have the following multiple sets of sentences:
\begin{itemize}
    \item Original set: the extracted sentences, as shown in Table \ref{tab:mustshe_sentences}. 
    \item Unmasked word sets: $5*|words|$ unmasked sentences for each sentences ($|words|$ denotes the number of words in the sentence)
\end{itemize}

\paragraph{Manual Replacement} Since very often the ambiguous words in the original sentences are unmasked with ambiguous words as well, we try to mitigate this by manually replacing the ambiguous words with the following non-ambiguous words: \textit{man, woman, girl, guy, boy}. We compare this approach with the originally replaced words by the BERT model.


%%%%%%%%%%%%%%%%%%%%%%%%%%%%%%%%%%%%%%%%%%%%%%%%%%%%%%%%%%%%%%%%%%%%%%%%%%%%%%%%%%%%%%%%%%%%
\section{Translation}
\label{sec:Real_Experiment:Translation}

We translate the sets of sentences from English to German in the two steps, also outlined in Section \ref{sec:Base_Experiment:Translation}:

\begin{enumerate}
    \item \textbf{Translation Source -> Target:} Translate the sets in the target language (German).
    \item \textbf{Backtranslation Target -> Source:} Translate the translations back into the source language (English).
\end{enumerate}

For translation, we use the Beam search decoding strategy (see Subsection \ref{sec:Background:Decoding}) with beam size 10.

%%%%%%%%%%%%%%%%%%%%%%%%%%%%%%%%%%%%%%%%%%%%%%%%%%%%%%%%%%%%%%%%%%%%%%%%%%%%%%%%%%%%%%%%%%%%
\section{Evaluation}
\label{sec:Real_Experiment:Evaluation}

For each original sentence, we execute the following algorithm:
\begin{enumerate}
    \item[1. ] Count the number of unique sentences in the backtranslations. \\
    $N \leftarrow |unique \; backtranslations|$ 
    \item[2. ] Count the number of unique sentences in the backtranslations for each unmasked word set of sentences. \\
    $[w_{11}, w_{21}, ..., w_{|words|1}]$, $[w_{12}, w_{22}, ...,  w_{|words|2}]$, $[w_{13}, w_{23}, ..., w_{|words|3}]$ 
    \item[3. ] Average the result for the five masks of each word. \\ % list of size |words|
    $[\frac{w_{11} + w_{12} + w_{13}}{5}, \frac{w_{21} + w_{22} + w_{22}}{5}, ...,  \frac{w_{|words|1} + w_{|words|2} + w_{|words|3}}{5}] = [w_{1}^{'}, w_{2}^{'}, ..., w_{|words|}^{'}]$
    \item[4. ] Subtract the number of unique backtranslations of the original sentence from the average.
    $[w_{1}^{'} - N, w_{2}^{'} - N, ..., w_{|words|}^{'} - N]$ 
    \item[5.] Extract the words, which generate the 5 biggest differences. 
    
\end{enumerate}


The extracted words indicate the 5 most ambiguous words in the original sentence.
% hypothesis

%%%%%%%%%%%%%%%%%%%%%%%%%%%%%%%%%%%%%%%%%%%%%%%%%%%%%%%%%%%%%%%%%%%%%%%%%%%%%%%%%%%%%%%%%%%%
\section{Results}
\label{sec:Real_Experiment:Results}


% add 5 best to table; ? maybe horizontal table


% Compared initial replacement with manual replacement
\chapter{Discussion}
\label{ch:Discussion}

% Compare original test set with disambiguated test set:
% - How different are translations in each nbest backtranslation list?
% - Are they more different for sentences with unresolved ambiguity?

%%%%%%%%%%%%%%%%%%%%%%%%%%%%%%%%%%%%%%%%%%%%%%%%%%%%%%%%%%%%%%%%%%%%%%%%%%%%%%%%%%%%%%%%%%%%
\section{Summary of Results}
\label{sec:Discussion:Summary}


%%%%%%%%%%%%%%%%%%%%%%%%%%%%%%%%%%%%%%%%%%%%%%%%%
\subsection{Intraset Evaluation}
\label{sec:Discussion:Intraset}

%%%%%%%%%%%%%%%%%%%%%%%%%%%%%%%%%%%%%%%%%%%%%%%%%
\subsection{Interset Evaluation}
\label{sec:Discussion:Interset}


%%%%%%%%%%%%%%%%%%%%%%%%%%%%%%%%%%%%%%%%%%%%%%%%%%%%%%%%%%%%%%%%%%%%%%%%%%%%%%%%%%%%%%%%%%%%
\section{Answers to Research Questions}
\label{sec:Discussion:Answers}




%%%%%%%%%%%%%%%%%%%%%%%%%%%%%%%%%%%%%%%%%%%%%%%%%%%%%%%%%%%%%%%%%%%%%%%%%%%%%%%%%%%%%%%%%%%%
\section{Challenges and Limitations}
\label{sec:Discussion:Challenges}

% - Word alignment methods introduce errors, which influence the results
% - Not always having both genders present in the translation nbest list -> beam 100
% - Using “male” vs “female” for disambiguation yields different results
% -- NMT models bias leads to translation effects which in turn influence our method



% ? Subsection about open/unresolved questions
\chapter{Conclusion and Future Work}
\label{ch:Conclusion}

% take ideas for Future work from literature reviews and thesis

% not only gender bias 

% The study could be extended to general ambiguity instead of only gender ambiguity.

%% --------------------
%% |   Bibliography   |
%% --------------------

%% Add entry to the table of contents for the bibliography
\printbibliography[heading=bibintoc]

%% ----------------
%% |   Appendix   |
%% ----------------
\appendix
\iflanguage{english}
{\chapter{Appendix}}    % english style
{\chapter{Anhang}}      % german style
\label{chap:appendix}

%%%%%%%%%% Uniqueness distribution %%%%%%%%%%%%
\begin{figure}[!htb]
     \centering
     
     \begin{subfigure}{0.49\textwidth}
         \centering
         \includegraphics[width=\textwidth]{figures/uniqueness/unique_beam100/unique_back_original.png}
         \caption{Ambiguous Subset}
         %\label{fig:uniqueness_ambiguous}
     \end{subfigure}
     \hfill
     \begin{subfigure}{0.49\textwidth}
         \centering
         \includegraphics[width=\textwidth]{figures/uniqueness/unique_beam100/unique_back_male.png}
         \caption{Disambiguated Subset (male)}
         %\label{fig:uniqueness_male}
     \end{subfigure}
     \begin{subfigure}{0.49\textwidth}
         \centering
         \includegraphics[width=\textwidth]{figures/uniqueness/unique_beam100/unique_back_average.png}
         \caption{Non-ambiguous Subset Average}
         %\label{fig:uniqueness_common}
     \end{subfigure}
     \hfill
     \begin{subfigure}{0.49\textwidth}
         \centering
         \includegraphics[width=\textwidth]{figures/uniqueness/unique_beam100/unique_back_female.png}
         \caption{Disambiguated Subset (female)}
         %\label{fig:uniqueness_female}
     \end{subfigure}
        \caption{Distribution of Unique Backtranslations: Beam search with beam size 100}
        \label{fig:uniqueness_graphs_100}

\end{figure}

\begin{figure}[!htb]
     \centering
     
     \begin{subfigure}{0.49\textwidth}
         \centering
         \includegraphics[width=\textwidth]{figures/uniqueness/unique_sampling/unique_back_original.png}
         \caption{Ambiguous Subset}
         %\label{fig:uniqueness_ambiguous}
     \end{subfigure}
     \hfill
     \begin{subfigure}{0.49\textwidth}
         \centering
         \includegraphics[width=\textwidth]{figures/uniqueness/unique_sampling/unique_back_male.png}
         \caption{Disambiguated Subset (male)}
         %\label{fig:uniqueness_male}
     \end{subfigure}
     \begin{subfigure}{0.49\textwidth}
         \centering
         \includegraphics[width=\textwidth]{figures/uniqueness/unique_sampling/unique_back_average.png}
         \caption{Non-ambiguous Subset Average}
         %\label{fig:uniqueness_common}
     \end{subfigure}
     \hfill
     \begin{subfigure}{0.49\textwidth}
         \centering
         \includegraphics[width=\textwidth]{figures/uniqueness/unique_sampling/unique_back_female.png}
         \caption{Disambiguated Subset (female)}
         %\label{fig:uniqueness_female}
     \end{subfigure}
        \caption{Distribution of Unique Backtranslations: Sampling}
        \label{fig:uniqueness_graphs_sampling}

\end{figure}


%%%%%%%%%% Alignment distribution %%%%%%%%%%%%

% Beam 100
% Distribution for Translation
\begin{figure}[!htb]
     \centering
     
     \begin{subfigure}{0.49\textwidth}
         \centering
         \includegraphics[width=\textwidth]{figures/alignment/align_100/word_translations_original.png}
         \caption{Ambiguous Subset}
         %\label{fig:alignment_translation_ambiguous}
     \end{subfigure}
     \hfill
     \begin{subfigure}{0.49\textwidth}
         \centering
         \includegraphics[width=\textwidth]{figures/alignment/align_100/word_translations_male.png}
         \caption{Disambiguated Subset (male)}
         %\label{fig:alignment_translation_male}
     \end{subfigure}
     \begin{subfigure}{0.49\textwidth}
         \centering
         \includegraphics[width=\textwidth]{figures/alignment/align_100/word_translations_average.png}
         \caption{Non-ambiguous Subset Average}
         %\label{fig:alignment_translation_common}
     \end{subfigure}
     \hfill
     \begin{subfigure}{0.49\textwidth}
         \centering
         \includegraphics[width=\textwidth]{figures/alignment/align_100/word_translations_female.png}
         \caption{Disambiguated Subset (female)}
         %\label{fig:alignment_translation_female}
     \end{subfigure}
        \caption[Distribution of Unique Translations for Words: Beam search with beam size 100]{\textbf{Distribution of Unique Translations for Words}. Beam search with beam size 100. Nbest size 100. Alignment with \textit{awesome-align}. The dashed line marks the average number of unique translations for the source word, the value displayed to the right.}
        \label{fig:alignment_graphs_translation_100}

\end{figure}

% Distribution for Backtranslation
\begin{figure}[!htb]
     \centering
     
     \begin{subfigure}{0.49\textwidth}
         \centering
         \includegraphics[width=\textwidth]{figures/alignment/align_100/word_backtranslations_original.png}
         \caption{Ambiguous Subset}
         %\label{fig:alignment_backtranslation_ambiguous}
     \end{subfigure}
     \hfill
     \begin{subfigure}{0.49\textwidth}
         \centering
         \includegraphics[width=\textwidth]{figures/alignment/align_100/word_backtranslations_male.png}
         \caption{Disambiguated Subset (male)}
         %\label{fig:alignment_backtranslation_male}
     \end{subfigure}
     \begin{subfigure}{0.49\textwidth}
         \centering
         \includegraphics[width=\textwidth]{figures/alignment/align_100/word_backtranslations_average.png}
         \caption{Non-ambiguous Subset Average}
         %\label{fig:alignment_backtranslation_common}
     \end{subfigure}
     \hfill
     \begin{subfigure}{0.49\textwidth}
         \centering
         \includegraphics[width=\textwidth]{figures/alignment/align_100/word_backtranslations_female.png}
         \caption{Disambiguated Subset (female)}
         %\label{fig:alignment_backtranslation_female}
     \end{subfigure}
        \caption[Distribution of Unique Backtranslations for Words: Beam search with beam size 100]{\textbf{Distribution of Unique Backtranslations for Words}. Beam search with beam size 100. Nbest size 100. Alignment with \textit{awesome-align}. The dashed line marks the average number of unique translations for the source word, the value displayed to the right.}
        \label{fig:alignment_graphs_backtranslation_100}

\end{figure}

% Sampling
% Distribution for Translation
\begin{figure}[!htb]
     \centering
     
     \begin{subfigure}{0.49\textwidth}
         \centering
         \includegraphics[width=\textwidth]{figures/alignment/align_sampling/word_translations_original.png}
         \caption{Ambiguous Subset}
         %\label{fig:alignment_translation_ambiguous}
     \end{subfigure}
     \hfill
     \begin{subfigure}{0.49\textwidth}
         \centering
         \includegraphics[width=\textwidth]{figures/alignment/align_sampling/word_translations_male.png}
         \caption{Disambiguated Subset (male)}
         %\label{fig:alignment_translation_male}
     \end{subfigure}
     \begin{subfigure}{0.49\textwidth}
         \centering
         \includegraphics[width=\textwidth]{figures/alignment/align_sampling/word_translations_average.png}
         \caption{Non-ambiguous Subset Average}
         %\label{fig:alignment_translation_common}
     \end{subfigure}
     \hfill
     \begin{subfigure}{0.49\textwidth}
         \centering
         \includegraphics[width=\textwidth]{figures/alignment/align_sampling/word_translations_female.png}
         \caption{Disambiguated Subset (female)}
         %\label{fig:alignment_translation_female}
     \end{subfigure}
        \caption[Distribution of Unique Translations for Words: Sampling]{\textbf{Distribution of Unique Translations for Words}. Sampling. Nbest size 10. Alignment with \textit{awesome-align}. The dashed line marks the average number of unique translations for the source word, the value displayed to the right.}
        \label{fig:alignment_graphs_translation_sampling}

\end{figure}

% Distribution for Backtranslation
\begin{figure}[!htb]
     \centering
     
     \begin{subfigure}{0.49\textwidth}
         \centering
         \includegraphics[width=\textwidth]{figures/alignment/align_sampling/word_backtranslations_original.png}
         \caption{Ambiguous Subset}
         %\label{fig:alignment_backtranslation_ambiguous}
     \end{subfigure}
     \hfill
     \begin{subfigure}{0.49\textwidth}
         \centering
         \includegraphics[width=\textwidth]{figures/alignment/align_sampling/word_backtranslations_male.png}
         \caption{Disambiguated Subset (male)}
         %\label{fig:alignment_backtranslation_male}
     \end{subfigure}
     \begin{subfigure}{0.49\textwidth}
         \centering
         \includegraphics[width=\textwidth]{figures/alignment/align_sampling/word_backtranslations_average.png}
         \caption{Non-ambiguous Subset Average}
         %\label{fig:alignment_backtranslation_common}
     \end{subfigure}
     \hfill
     \begin{subfigure}{0.49\textwidth}
         \centering
         \includegraphics[width=\textwidth]{figures/alignment/align_sampling/word_backtranslations_female.png}
         \caption{Disambiguated Subset (female)}
         %\label{fig:alignment_backtranslation_female}
     \end{subfigure}
        \caption[Distribution of Unique Backtranslations for Words: Sampling]{\textbf{Distribution of Unique Backtranslations for Words}. Sampling. Nbest size 10. Alignment with \textit{awesome-align}. The dashed line marks the average number of unique translations for the source word, the value displayed to the right.}
        \label{fig:alignment_graphs_backtranslation_sampling}

\end{figure}

\end{document}
