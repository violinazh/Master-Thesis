\chapter{Experimental Setup}
\label{ch:Setup}

In this chapter, we outline the setup for the experiments and describe the used datasets, models, evaluation methods and tools.

%%%%%%%%%%%%%%%%%%%%%%%%%%%%%%%%%%%%%%%%%%%%%%%%%%%%%%%%%%%%%%%%%%%%%%%%%%%%%%%%%%%%%%%%%%%%
\section{Languages}
\label{sec:Setup:Languages}

% Source language: English 
% Target languages:
% - High resource: German (Germanic), French (Romance)
% - Low resource: Bulgarian (Slavic)

%%%%%%%%%%%%%%%%%%%%%%%%%%%%%%%%%%%%%%%%%%%%%%%%%
\subsection{Source Language} 
The source language used for translation is English, because translating from a notional gender language (English) into a grammatical gender language (e.g., German, French, Bulgarian) can produce biases by translating a non-gendered noun into the wrong gendered noun due to an unjustified assumption. Also, English is the most spoken language in the world, and many of the existing large datasets for training NMT models have English as either their source or target language.

%%%%%%%%%%%%%%%%%%%%%%%%%%%%%%%%%%%%%%%%%%%%%%%%%
\subsection{Target Language} 
The main target language for the base experiments of this study is German. This is primarily due to the writer's knowledge of the language, which leads to easier manual evaluation when necessary, as well as because the main occupational dataset, WinoMT \parencite{Stanovsky_2019}, also has gender evaluation for German.

% The experiments will be extended to other language families and lower-resource languages such as Bulgarian.

%%%%%%%%%%%%%%%%%%%%%%%%%%%%%%%%%%%%%%%%%%%%%%%%%%%%%%%%%%%%%%%%%%%%%%%%%%%%%%%%%%%%%%%%%%%%
\section{Datasets}
\label{sec:Setup:Datasets}

% - test dataset for general MT model performance
% - test set (challenge set or natural corpora) for assessing gender bias

We use two types of datasets — challenge sets and natural corpora. Challenge sets are synthetically created sentences, designed to be used in a controlled experiment environment to evaluate a specific phenomenon. In contrast, natural corpora are comprised of naturally occurring sentences that are meant for training and testing phenomena in real-world scenarios.

%%%%%%%%%%%%%%%%%%%%%%%%%%%%%%%%%%%%%%%%%%%%%%%%%
\subsection{Challenge Test Set}
\label{sec:Setup:Challenge_Set}

For detecting gender ambiguous words, we used the tagged challenge test set WinoMT, developed by \citet{Stanovsky_2019}. It consists of 3888 synthetic sentences presenting two human entities defined by their occupation and a subsequent pronoun that needs to be correctly resolved to match the gender of one of the entities. It also contains an equal balance between male and female gender nouns, as well as between stereotypical and non-stereotypical gender-role assignments (e.g., a female doctor versus a female nurse). We can see in Table \ref{tab:winomt} two sentences of the original dataset.

% - Downsides: synthetic samples - controlled experiment environment, but may introduce some artificial biases; only English as source language; too small set for training easy to overfit

\begin{table}
    \label{tab:winomt}
    \begin{tabularx}{\linewidth}{|X|l|l|l|}
        \hline
        \textbf{Source Sentence} & \textbf{Ambiguous word} & \textbf{Position} & \textbf{Gender} \\ \hline
        The \textbf{developer} argued with the designer because she did not like the design. & developer & 1 & female \\ \hline
        The developer argued with the \textbf{designer} because his idea cannot be implemented. & designer & 5 & male \\ \hline
    \end{tabularx}
    \caption{Example: WinoMT challenge set}
\end{table}

%%%%%%%%%%%%%%%%%%%%%%%%%%%%%%%%%%%%%%%%%%%%%%%%%
\subsection{Natural Corpora}
\label{sec:Setup:Natural_Corpora}

In order to evaluate the approach in a natural setting, we used the natural multilingual corpus MuST-SHE \parencite{MuST-SHE}, designed to evaluate the performance of NMT systems in the translation of gender for English to Spanish/French/Italian. It comprises naturally occurring instances of spoken language retrieved from MuST-C \parencite{MuST-C}, which is built on TED Talks data. The samples in the dataset are balanced between masculine and feminine phenomena. They include sentences representing four different types of gender phenomena, which are classified based
on the type of information needed to disambiguate gender translation. We are specifically interested in the fourth category, which comprises sentences, for which no gender-disambiguating information can be retrieved from context, referred previously as \textit{unresolvable ambiguity}. It contains 34 sentences in total. In Table \ref{tab:mustshe} we can see two sentences of the category. 

\begin{table} 
    \label{tab:mustshe}
    \begin{tabularx}{\linewidth}{|X|l|l|}
        \hline
        \textbf{Source Sentence} & \textbf{Ambiguous Word} & \textbf{Gender} \\ \hline
        We have our cognitive biases, so that I can take a perfect history on a \textbf{patient} with chest pain. & patient & male \\ \hline
        And there was perpetual \textbf{victim} blaming when these victims came to report their crimes. & victim & female \\ \hline
    \end{tabularx}
    \caption{Example: MuST-SHE dataset}
\end{table}

%%%%%%%%%%%%%%%%%%%%%%%%%%%%%%%%%%%%%%%%%%%%%%%%%%%%%%%%%%%%%%%%%%%%%%%%%%%%%%%%%%%%%%%%%%%%
\section{NMT Models}
\label{sec:Setup:Models}

We use the following pre-trained NMT models:

\subsection{English <-> German}
For translating into German, we rely on Fairseq's \footnote{https://github.com/facebookresearch/fairseq/blob/main/examples/wmt19/README.md} WMT’19 ensemble Transformer model, developed by Facebook \parencite{WMT19}. WMT is a MT dataset composed of a collection of various sources, including news commentaries and parliament proceedings. The corpus file has around 4M sentences, translated by professional translators.

% - French-English WMT’14 model, finetuned on MuST-C data to use for MuST-SHE evaluation

%%%%%%%%%%%%%%%%%%%%%%%%%%%%%%%%%%%%%%%%%%%%%%%%%%%%%%%%%%%%%%%%%%%%%%%%%%%%%%%%%%%%%%%%%%%%
\section{Evaluation Methods}
\label{sec:Setup:Evaluation}

% TODO
% Evaluation procedures ought to cover both models’ general performance and gender-related issues. This is crucial to establish the capabilities and limits of mitigating strategies \cite{Savoldi_2021}.

% - Model performance metric: BLEU (translation quality)
% - Gender accuracy metric (gender bias)

%%%%%%%%%%%%%%%%%%%%%%%%%%%%%%%%%%%%%%%%%%%%%%%%%%%%%%%%%%%%%%%%%%%%%%%%%%%%%%%%%%%%%%%%%%%%
\section{Technical Details}
\label{sec:Setup:Technical}

%%%%%%%%%%%%%%%%%%%%%%%%%%%%%%%%%%%%%%%%%%%%%%%%%
\subsection{Tools}
\label{sec:Experiments:Tools}

In this section, we mention the most used tools during the development of the experiments. The programming code for the experiments is written in Python, and we use PyTorch's library for deep learning.

\paragraph{Pre-processing Tools:}
\begin{itemize}
    \item \textbf{Sacremoses \footnote{https://github.com/alvations/sacremoses}:} A pre-porcessing tool for the tokenization and de-tokenization of text.
    \item \textbf{Subword NMT \footnote{https://github.com/rsennrich/subword-nmt}:} A pre-processing tool for segmenting text into subword units. % used for the French models
    \item \textbf{Fast BPE \footnote{https://github.com/glample/fastBPE}:} A pre-processing tool for segmenting text into subword units. % used for the German models
    \item \textbf{Spacy \footnote{https://spacy.io/}:} An open-source library for Natural Language Processing. We use it for the detection of stop words, lemmatization and prediction of gender.
\end{itemize}

\paragraph{Word Alignment Tools:}
\begin{itemize}
    \item \textbf{Fast align \footnote{https://github.com/clab/fast\_align}:} A simple and fast unsupervised word aligner developed by \citet{fast-align}. This tool is used to align the translated sentences with the source sentences.
    \item \textbf{Awesome align \footnote{https://github.com/neulab/awesome-align}:} A contextualized embedding-based word aligner that extracts word alignments based on similarities of the tokens’ contextualized embeddings \parencite{awesome-align}. Awesome-align achieves state-of-the-art performance on five language pairs. It can extract word alignments from multilingual BERT (mBERT) and can be fine-tuned on parallel corpora for better alignment quality. We use the base mBERT model provided in the package to align the source sentences with the translated sentences.
    \item \textbf{Tercom alignment \footnote{https://github.com/jhclark/tercom}:} A tool for aligning between different translations of the source sentence. We use it to align the source sentences with the corresponding backtranslations.
\end{itemize}

%%%%%%%%%%%%%%%%%%%%%%%%%%%%%%%%%%%%%%%%%%%%%%%%%
\subsection{Hardware Resources}
\label{sec:Setup:Hardware}

% TODO
% - GPU: GeForce GTX 1080 Ti
% - Batch size
% - Training time
