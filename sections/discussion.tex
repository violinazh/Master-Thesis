\chapter{Discussion}
\label{ch:Discussion}

This chapter presents a summary of results from conducting the experiments and discusses challenges and limitations of the approach. 

% Compare original test set with disambiguated test set:
% - How different are translations in each nbest backtranslation list?
% - Are they more different for sentences with unresolved ambiguity?

%%%%%%%%%%%%%%%%%%%%%%%%%%%%%%%%%%%%%%%%%%%%%%%%%%%%%%%%%%%%%%%%%%%%%%%%%%%%%%%%%%%%%%%%%%%%
\section{Base Experiment}
\label{sec:Discussion:Base}

In our base experiment, we constructed four subsets to inspect the initial assumption, that sentences containing an ambiguity produce less diverse backtranslations than sentences without an ambiguity. We extracted fully ambiguous sentences from the synthetic dataset WinoMT. In one approach, we attempted disambiguating the source ambiguous words in the sentences with gender prefix words: \textit{male} and \textit{female}. Another approach consisted of replacing the source ambiguous words with five common words (\textit{man, woman, girl, guy, boy}) and averaging the results. We translated and backtranslated the subsets and evaluated the results based on gender, uniqueness of translations and backtranslations, reoccurrence of the source sentence and source word and translations of the source word and words in the rest of the sentence.

%%%%%%%%%%%%%%%%%%%%%%%%%%%%%%%%%%%%%%%%%%%%%
\subsection{Replacement Method} % Disambiguation vs. common words 
The evaluation of the results showed that disambiguating with \textit{male} and \textit{female} has different effects on the translation and backtranslation. The female-disambiguated subsets often lost the \textit{female} prefix in backtranslation due to not directly translating it to the corresponding German word, but directly producing the female gender noun in translation. On the other hand, the male-disambiguated subset most often kept the \textit{male} prefix in translation and backtranslation. From this, we can conclude that disambiguating with \textit{female} proved to be more successful in terms of translation quality.

The replacement of an ambiguous word using disambiguation proved successful when using Beam search with beam size 100 and Sampling for generating translations. The ambiguous subset produced less unique backtranslations, proving Hyp. \nameref{a}) and \nameref{c}). Also, the ambiguous word in the ambiguous subset reoccurred most often in backtranslation, confirming Hyp. \nameref{d}).

Furthermore, when we only regard the ambiguous and the female-disambiguated subsets for Beam search 
with beam size 10, we observe that the ambiguous subset does produce less unique sentences in the backtranslations as well as less unique backtranslations of the source word, which partially confirms the initial Hypothesis \nameref{main}. Since we showed that disambiguating with \textit{female} produces better quality translations than disambiguating with \textit{male}, this result is positive.

On the other hand, replacing the ambiguous word with a common unambiguous word proved mostly ineffective for all search strategies. From the results on uniqueness of the backtranslations, we observed that the ambiguous subset generates the least unique sentences in the backtranslations compared to the disambiguated subsets and the average of the non-ambiguous subset, both for beam size 10 and 100. However, when comparing the result of the ambiguous subset against the result for the common words in the non-ambiguous subset individually, we noticed that there is a lower value. This could be due to the chosen words of replacement. It may mean that the occupational ambiguous words have more semantic meanings outside the ambiguity of gender. From the analysis on alignment, we saw that the occupation words have multiple versions in both the source and the target language. For example, the word \textit{man} most often gets translated to the single word “Mann”, while the gender ambiguous word \textit{developer} has been translated to more than three different words not considering gender information: “Bauträger”, “Bauunternehmer”, “Entwickler”. This in turn influences the backtranslation, which also produces more unique backtranslations of the sentences. This can also be observed in the results from the alignment, where the average results for the non-ambiguous subset has the lowest value in unique translations of the source word, even when disregarding gender information for the ambiguous subset. We can conclude that in this case, the multiple meanings of the word have stronger influence on the diversity of translation than its gender ambiguity. Therefore, the replacement method using common words can be deemed unsuccessful in this scenario.

%%%%%%%%%%%%%%%%%%%%%%%%%%%%%%%%%%%%%%%%%%%%%
\subsection{Search Method} % Beam search vs. Sampling; Beam size 10 vs. 100

We decided to compare Beam search with Sampling to see what effect it would have on the diversity of translation. Also, \citet{roberts2020decoding} prove that Beam search unlike Sampling has a tendency to generate more frequent masculine pronouns, because it guides the model towards an extreme operating point that exhibits zero variability and consequently amplifies the biases present in the training data, resulting in a skewed output. Indeed, the Sampling method introduced more variability regarding gender. Furthermore, it led to more unique words and sentences in backtranslation overall. 

While for Beam search with beam size 10 the ambiguous word in the ambiguous subset has more unique backtranslations than the male-disambiguated subset, decoding with Beam search with beam size 100 or Sampling counteracts that and shows that the ambiguous word has the least amount of backtranslations compared to the disambiguated subset, confirming Hyp. \nameref{c}).


%%%%%%%%%%%%%%%%%%%%%%%%%%%%%%%%%%%%%%%%%%%%%
\subsection{Correlation} 

Furthermore, we detected a possible correlation between the number of translations and backtranslations. The number of backtranslations seem to increase proportionally to the number of translations for all subsets.

In regard to the alignment results, we observed that the rest of the sentence produces the most unique translations and backtranslations for the non-ambiguous subset. But most importantly, the difference in the scores for the source word and the rest of the sentence is a lot bigger for the ambiguous subset than the non-ambiguous subset. This proves that the most diversity in translation is given to the ambiguous word in the sentence, when such is present. Therefore, we can conclude that the number of translations of the non-ambiguous words in a sentence is correlated with the number of translations of the ambiguous word in the sentence.


% %%%%%%%%%%%%%%%%%%%%%%%%%%%%%%%%%%%%%%%%%%%%%%%%%
% \subsection{Intraset Evaluation}
% \label{sec:Discussion:Intraset}

% %%%%%%%%%%%%%%%%%%%%%%%%%%%%%%%%%%%%%%%%%%%%%%%%%
% \subsection{Interset Evaluation}
% \label{sec:Discussion:Interset}

 
%%%%%%%%%%%%%%%%%%%%%%%%%%%%%%%%%%%%%%%%%%%%%%%%%%%%%%%%%%%%%%%%%%%%%%%%%%%%%%%%%%%%%%%%%%%%
\section{Real-world Experiment}
\label{sec:Discussion:Real}

The real-world experiment aimed to probe the assumption that sentences containing ambiguous words generate less unique backtranslations, and with this to attempt detecting ambiguous words in a realistic sentence. We developed an algorithm of replacing each word in the sentence with the most probable word, unmasked with a BERT model, while also manually replacing the gender ambiguous nouns with non-ambiguous words. We concluded from the results that the method of combining BERT with manual replacement of the ambiguous words appears more effective than the simple replacement method with BERT, proving that replacing ambiguous words with non-ambiguous words in a sentence may assist in detecting ambiguous words.


%%%%%%%%%%%%%%%%%%%%%%%%%%%%%%%%%%%%%%%%%%%%%%%%%%%%%%%%%%%%%%%%%%%%%%%%%%%%%%%%%%%%%%%%%%%%
\section{Challenges and Limitations}
\label{sec:Discussion:Challenges}

While conducting the experiments and evaluating the results, we also met a couple of challenges.

\paragraph{Gender Bias}
Pre-trained NMT models are gender biased, which influences the balance of male and female translations. For Beam search with beam size 10, we observed that in less than half of the translations, both genders occur, which influences the way of evaluating the assumption. Furthermore, we observed from the results on gender that there is an inclined tendency of translation to produce more male nouns than female. To combat these problems, we applied Beam search with beam size 100. This successfully contributed to both genders occurring in over 90\% of the translations, however it only slightly improved the balance between male and female translations of the source word. It also carries specific drawbacks, such as worsening the translation quality overall. Sampling also slightly counteracts this issue, but it is still insufficient.

Furthermore, we noticed that disambiguating with \textit{male} versus \textit{female} yields different results. Unexpectedly, sometimes the gender words were completely disregarded, and the wrong gender noun was produced in translation. This suggests that the method of disambiguation using gender forcing may not be very effective.

\paragraph{Word Alignment}
Another challenge we encountered relates to the word alignment methods. Often, \textit{fast-align} and \textit{awesome-align} produced significantly different results, although the results did not differ in terms of the end conclusion towards the initial assumption. We eventually relied mostly on the results from \textit{awesome-align}, because it is state-of-the-art. It would also be useful to evaluate the quality of the alignment methods, which we didn't specifically investigate due to time constraints.

\paragraph{Translation Quality}
The quality of translation presented itself to be another limitation, because the WinoMT dataset does not provide reference translations currently, therefore only manual assessment was possible. An automatic assessment would have been useful for the case, when we explored translation with the Beam search algorithm with beam size 100 to inspect the difference of quality within the nbest list.

\newpage

\paragraph{Ambiguity Definition}
When detecting ambiguous words, we need to take into account the type of ambiguity we are testing for. Since we defined ambiguity as having one version in the source language and multiple versions in the target language, the occupational words are gender-ambiguous in this definition, but they are not generally ambiguous. However, they have multiple semantic meanings in both chosen source and target language. Therefore, replacing them with words, which have less semantic meanings in both languages (e.g., \textit{man, woman, girl, guy, boy}), proved not useful for assessing their gender ambiguity.

\paragraph{Real-world Scenario}
In regard to detecting ambiguous words in a real-world setting, we also met a couple of challenges. The method of replacement using BERT is not always successful in replacing the ambiguous words with non-ambiguous words, which facilitated the need for manual replacement. This manual replacement would only be applicable in the small set of sentences we use for proof-of-concept, but not in a large dataset. Therefore, we outline the need for an automatic method of replacement of ambiguous words with non-ambiguous words. 

Furthermore, the marked ambiguous words in the sentences were only gender-ambiguous nouns. However, there may be other detected ambiguous words in these sentences, which were not marked. We do not limit our approach to only detecting gender ambiguous words, but we currently have only the means to evaluate for such words due to the datasets we used.



