\chapter{Methodology}
\label{ch:Methodology}
% Most important chapter: talks about my contribution to the topic and what I have achieved

%%%%%%%%%%%%%%%%%%%%%%%%%%%%%%%%%%%%%%%%%%%%%%%%%%%%%%%%%%%%%%%%%%%%%%%%%%%%%%%%%%%%%%%%%%%%
\section{Problem Statement}
\label{sec:Methodology:Problem}

%%%%%%%%%%%%%%%%%%%%%%%%%%%%%%%%%%%%%%%%%%%%%%%%%%%%%%%%%%%%%%%%%%%%%%%%%%%%%%%%%%%%%%%%%%%%
\section{Approach}
\label{sec:Methodology:Approach}


Intuition/hypothesis/assumption: Ambiguous words generate less unique words in backtranslation than non-ambiguous words

% One method for detecting the biases term could be identifying terms that have multiple translations into one direction, but only one translation in the other. 


Goal: Detect ambiguous words in a sentence without context
	- Develop method(s) to differentiate from non-ambiguous words: uncover patterns in translation and backtranslation
	- Finetune parameters (e.g. how often ambiguous word reoccurs in backtranslation; how many unique words in translation vs. backtranslation)
	- Suggest alternative translations to ambiguous word
    - Differentiate ambiguous from biased words (cannot say if bias exists or not -> Quality Estimation)

% Disambiguation method (male vs. female)
% Word alignment statistics
