\Abstract

In der Ära der Globalisierung ist die Notwendigkeit einer effektiven Kommunikation in verschiedenen Sprachen von größter Bedeutung. Die neuronale maschinelle Übersetzung (NMT), ein Teilbereich des maschinellen Lernens, hat in diese Richtung bedeutende Fortschritte gemacht, indem sie Deep Learning verwendet, um direkt von einer Eingabesprache in eine Ausgabesprache zu übersetzen. Diese Modelle perpetuieren jedoch oft soziale Konstrukte und Stereotypen, die in den Trainingsdaten vorhanden sind, was zu voreingenommenen Übersetzungen führt.

Diese Arbeit stellt eine neuartige Methode zur Erkennung mehrdeutiger Wörter in Texten vor, die zu Vorurteilen führen könnten, indem die Vielfalt der Übersetzung untersucht wird. Wir stellen die Hypothese auf, dass Sätze, die mehrdeutige Wörter enthalten, die in der Quellsprache eine Version und in der Zielsprache mehrere Versionen haben, weniger vielfältige Rückübersetzungen erzeugen als Sätze ohne Mehrdeutigkeit. Um diese Hypothese zu testen, vergleichen wir einen Datensatz mit mehrdeutigen Sätzen mit einem solchen, bei dem das mehrdeutige Wort durch ein nicht mehrdeutiges Äquivalent ersetzt wird. Unser Ansatz offenbart Muster in der Übersetzungsdiversität, die auf Mehrdeutigkeit hindeuten könnten und bietet einen neuen Rahmen zur Erkennung von Mehrdeutigkeit in Texten, die keine kontextuellen Informationen zur Mehrdeutigkeit enthalten. Die Ergebnisse zeigen, dass das Ersetzen eines mehrdeutigen Wortes durch sein eindeutiges Äquivalent in einigen Szenarien zu vielfältigeren Übersetzungen führt und somit die Hypothese teilweise bestätigt. Darüber hinaus haben wir unseren Ansatz in einem Szenario aus der realen Welt mit natürlich gesprochenen Sätzen getestet, die Mehrdeutigkeiten enthalten. Das Experiment zeigte das Potenzial des Ansatzes zur Erkennung mehrdeutiger Wörter auf und deckte gleichzeitig spezifische Einschränkungen der Methode auf.

Diese Forschung hat bedeutende Auswirkungen auf das Gebiet der NMT und bietet eine potenzielle Lösung für das Problem ungerechtfertigter Annahmen über mehrdeutige Wörter, die zu Vorurteilen in maschinellen Übersetzungen führen. Durch das Aufdecken von Mehrdeutigkeiten können wir an genaueren und faireren Übersetzungen arbeiten und so eine bessere interkulturelle Kommunikation und Verständigung fördern.
