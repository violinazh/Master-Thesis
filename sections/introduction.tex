\chapter{Introduction}
\label{ch:Introduction}

% Title: Detecting Ambiguity in Neural Machine Translation Models by Inspecting Diversity in Translation

% PRICoBE:
% Problem: What is the problem you are trying to address?
% Research questions: What are the research questions you are trying to answer? (to make novelty/originality clear)
% Idea: What is your idea of how to address the problem? (is often broader than the contribution)
% Contributions: Which contributions to the research area are you making with that idea?
% Benefits: What are the benefits of your contributions?
% Evaluation: How do you plan to evaluate your envisioned benefits? (Validation)

% Define clear research objectives: Clearly define the goals and objectives of your research. These objectives should be specific, measurable, achievable, relevant, and time-bound (SMART).


% Introduce ML, MT, NMT
In the world more than 7000 natural languages are spoken nowadays. It is humanly impossible to learn every language, which outlines the need for translation between different languages for the purpose of communication. This is a task for Machine Translation (MT), a subfield of Machine Learning (ML) that focuses on automatic translation from one language to another using computer technology. Machine Translation (MT) technology has assisted humans in gathering, processing, and communicating information.  

%%%%%%%%%%%%%%%%%%%%%%%%%%%%%%%%%%%%%%%%%%%%%%%%%%%%%%%%%%%%%%%%%%%%%%%%%%%%%%%%%%%%%%%%%%%%
\section{Motivation}
\label{sec:Introduction:Motivation}

% Elaborate on the problem of bias and ambiguity (interplay of bias and ambiguity) and how it affects different individuals
The development of Artificial Intelligence (AI) in recent years has expanded the field of MT and made it possible for people from all over the world to connect, learn and work in a foreign language. One application of AI is Neural Machine Translation (NMT), which uses deep learning to learn a statistical neural model for machine translation in an end-to-end fashion, translating directly from an input source language to an output target language. Neural networks (NNs) are typically trained on large corpora of natural occurring data extracted from the internet \parencite{NMT}. 
One problem with this data is it often contains social constructs and stereotypes. As a consequence, NMT models learn the biases from the data and perpetuate them, affecting downstream applications like coreference resolution \parencite{Zhao_2018_coreference} and contributing further to discrimination based on gender, race, age and religious beliefs \parencite{Rudinger_2017}. Some examples of this phenomenon include under-representation of women, stereotyping professions, e.g., associating doctors with men and nurses with women \parencite{Escud_Font_2019}, and stereotyping behaviors, e.g., associating women with gossiping and men with guitars \parencite{Rudinger_2017}. Stereotypical assumptions in turn tend to impact individuals' perceptions of reality and influence their behavior in accordance with stereotypical expectations.

%%%%%%%%%%%%%%%%%%%%%%%%%%%%%%%%%%%%%%%%%%%%%%%%%%%%%%%%%%%%%%%%%%%%%%%%%%%%%%%%%%%%%%%%%%%%
\section{Research Questions}
\label{sec:Introduction:Questions}

% State objective
The objective of this work is to develop a method to detect ambiguous words in written text by inspecting the diversity of translation. In order to achieve this, I attempt to answer the following questions systematically.

\paragraph{Main Research Question: } How can we detect ambiguous words in translated written text?
\begin{itemize}
    \item How diverse are translations? % inspecting diversity patterns in translations which could point to ambiguity
    \item How does language influence the discovery of ambiguity? % different language families and alphabets may have different effect on translation ambiguous words
    \item How does context influence the discovery of ambiguity? 
\end{itemize}

% TODO: mention the type of ambiguity I am detecting (unresolved ambiguity, gender bias)

%%%%%%%%%%%%%%%%%%%%%%%%%%%%%%%%%%%%%%%%%%%%%%%%%%%%%%%%%%%%%%%%%%%%%%%%%%%%%%%%%%%%%%%%%%%%
\section{Contribution}
\label{sec:Introduction:Contribution}

As a part of the thesis, I want to contribute to solving the problem of bias in Machine Translation by developing an approach for detecting ambiguous words that could lead to bias. 
% TODO: Summarize approach, evaluation method and results 

In recent years, light has been shed on the different types of biases present in Neural Machine Translation (NMT) systems, the most researched type being gender bias \parencite{Savoldi_2021}. Some previous works have attempted to uncover gender bias in existing systems \parencite{Prates_2019}, while others have tried mitigating gender bias by either modifying the data (e.g, \citet{Escud_Font_2019}, \citet{Stanovsky_2019}) or changing the architecture of the system \parencite{Vanmassenhove_2018}. While there have been some studies on finding biases in MT, this is the first study aiming to create a framework for detecting ambiguity in a text, which contains no contextual information relating to the ambiguity, therefore making several translations possible. The ability to uncover ambiguity could in turn help to alleviate the problem of MT systems making an unjustified assumption, leading to bias.


%%%%%%%%%%%%%%%%%%%%%%%%%%%%%%%%%%%%%%%%%%%%%%%%%%%%%%%%%%%%%%%%%%%%%%%%%%%%%%%%%%%%%%%%%%%%
\section{Thesis Outline}
\label{sec:Introduction:Outline}
The rest of the thesis is structured as follows. Chapter \ref{ch:Background} describes the background of NMT systems and introduces the problems stemming from language ambiguity and bias in the data. Next, Chapter \ref{ch:Related_work} introduces some existing research on the topic, such as techniques to detect, assess, and mitigate bias in MT. Chapter \ref{ch:Methodology} states the research problem and describes the approach used to answer the research questions. Furthermore, Chapter \ref{ch:Setup} describes the design of the experiments performed, which includes the corpora, models and evaluation methods used to conduct the experiments as well as technical details necessary for reproducibility. Chapter \ref{ch:Experiments} describes the experiments, which were performed. In Chapter \ref{ch:Results} the results of the experiments are presented and discussed. Chapter \ref{ch:Discussion} outlines the answers to the research questions from conducting the experiments and discusses challenges and limitations. Finally, Chapter \ref{ch:Conclusion} summarizes the key findings and proposes possible directions for future work.